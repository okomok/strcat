\section{Preliminaries}

% \begin{definition}[Vacuous Truth]
% Given boolean expressions $P$ and $Q$, a boolean expression
% \In{align*}{
%     P \implies Q
% } 
% is true if $P$ is false.
% As a consequence,
% \In{align*}{
%     (\forall x \in X)(Q(x)) \mydefeq (\forall x)(x \in X \implies Q(x))
% }
% is true in case $X$ is the empty set.
% \end{definition}


\subsection{Universality}

\begin{definition}
For a boolean-valued function $P$, define
\In{align*}{
    !aP(a) & \mydefeq P(a) \wedge \forall a' (P(a') \implies a = a')
}
\end{definition}
\begin{definition}[Uniqueness Quantification]
Define
\In{align*}{
    \exists!aP(a) & \mydefeq \exists a ! a P(a) 
}
meaning that "there exists a unique $a$ such that $P$".
\end{definition}
\begin{remark}
On the other hand,
\In{align*}{
    \exists a \big((!a P(a)) \wedge Q(a)\big)
}
means "there exists a unique $a$ such that $P$, moreover the $a$ is $Q$".
\end{remark}

\begin{definition}[Universality]
We boldly call a statement of the form
\In{align*}{
    (\forall x \in X)(\exists!y\in Y)(P(x,y))
}
\emph{universality}.
\end{definition}

\begin{proposition}[Functional Universality]
\In{align*}{
    & (\forall x \in X)(\exists!y\in Y)(P(x,y)) \\
    \iff & (\exists f : X \to Y)(\forall x \in X)(\forall y \in Y)(P(x,y) \iff y = f(x))
}
\end{proposition}
\begin{strdproof}
$(\implies)$ is from the Axiom of choice. $(\impliedby)$ is immediate.
\end{strdproof}
\mynewline
Category theory is the study to build such a function $f$.

\begin{definition}[Bijectivity]
Given a a function $g : Y \to X$, a statement
\In{align*}{
    (\exists f : X \to Y)(\forall x \in X)(\forall y \in Y)(x = g(y) \iff y = f(x))
}
is known as the \emph{bijectivity} of $f$ and $g$.
\end{definition}
This is a special case of universality with $P(x,y)$ being $x = g(y)$.

\subsection{Lambda Expressions}

\begin{definition}[Lambda Expression]
Following famous symbols like $\Sigma$, define
\In{align*}{
    \Lambda_x y \mydefeq x \mapsto y 
}
for anonymous functions.
\end{definition}
\begin{definition}
Given a function $H$ whose domain is a set of functions, define
\In{align*}{
    H_x y \mydefeq H(\Lambda_x y)
}
\end{definition}

\begin{definition}[Placeholder Expression]
For simple lambda expressions, you may use \emph{placeholders}:
\In{align*}{
    \texttt{?}{+}1 \mydefeq \Lambda_n n{+}1
}
Placeholder symbols can vary: $\texttt{?}$, $\texttt{-}$, $\texttt{1}$, etc.
\end{definition}


\subsection{Families}
Syntax of the function application is world-standard and fixed:
\In{align*}{
    f(x) \text{ or } fx
}
, but sometimes you might want cuter syntax like that
\newcommand*{\myxstar}{%
    \raisebox{0.12ex}{\ensuremath{%
        \strdgathered{\strdcellof{strd cellbase,sharp corners,regular polygon,regular polygon sides=5}{x}}%
    }}%
}
\In{align*}{
    \myxstar
}
In such case,
\begin{definition}[Family Declaration]
A \emph{family declaration} is an easy way to define arbitary application syntax:
\In{align*}{
    (\myxstar \in Y) _ {x \in X}
}
\end{definition}
\begin{definition}[Family Definition]
If you place a function implementation into a family declaration:
\In{align*}{
    (\myxstar \mydefeq x^2 \in Y) _ {x \in X}
}
, it might be called a \emph{family definition}.
\end{definition}
Family declarations can do more. 
\begin{definition}[Dependent Function]
Let $F$ a set-valued function.
\In{align*}{
    (f (x) \in F(x)) _ {x \in X}
}
defines a function
\In{align*}{
    f : X \to \textstyle\bigcup _ {x \in X} F(x)
}
such that
\In{align*}{
    (\forall x \in X)(f(x) \in F(x))
}
Such $f$ is called a \emph{dependent function}, for the $F(x)$ depends on $x$. 
\end{definition}
In case $F$ is a constant function,
\In{align*}{
    (f(x) \in Y)_{x \in X}
}
is a normal function $X \to Y$. 

\subsection{Coherence}
You will write
\In{align*}{
    3+1+2
}
rather than
\In{align*}{
    (3+(0+1))+2
}
because you know the arithmetic laws
\In{align*}{
    x+(y+z) = (x+y)+z \\
    0+x = x = x+0
}
disambiguate unparenthesized expressions.
Informally a set of laws to introduce simpler syntax is called a \emph{coherence condition} or briefly \emph{coherence}.

