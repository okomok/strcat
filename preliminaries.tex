\section{Preliminaries}

\subsection{Vacuous Truth}

Given boolean expressions $P$ and $Q$, a boolean expression
\In{align*}{
    P \implies Q
} 
is true if $P$ is false.
As a consequence,
\In{align*}{
    (\forall x \in X)(Q(x)) \mydefeq (\forall x)(x \in X \implies Q(x))
}
is true in case $X$ is the empty set.

\subsection{Uniqueness Quantification}

For a boolean-valued function $P$, define
\In{align*}{
    !aP(a) & \mydefeq P(a) \wedge \forall a' (P(a') \implies a = a')
}
You can state "there exists a unique $a$ such that $P$" using
\In{align*}{
    \exists!aP(a) & \mydefeq \exists a ! a P(a) 
}
On the other hand,
\In{align*}{
    \exists a \big((!a P(a)) \wedge Q(a)\big)
}
means "there exists a unique $a$ such that $P$, moreover the $a$ is $Q$".

\subsection{Universality}

Category theory introduces many \textit{universal properties} or \textit{universality} like 
\In{align*}{
    (\forall x \in X)(\exists!y\in Y)(P(x,y))
}
, which is equivalent to an easier form:
\In{align*}{
    (\exists f : X \to Y)(\forall x \in X)(\forall y \in Y)(P(x,y) \iff y = f(x))
}
Category theory is the study to build such a function $f$.
\mynewline
In case $P(x,y)$ is $x = g(y)$ with a function $g : Y \to X$,
\In{align*}{
    (\exists f : X \to Y)(\forall x \in X)(\forall y \in Y)(x = g(y) \iff y = f(x))
}
is known as \textit{bijectivity}.

\subsection{Lambda Expressions}

Following famous symbols like $\Sigma$, define
\In{align*}{
    \Lambda_x y \mydefeq x \mapsto y 
}
for anonymous functions.

\subsubsection{Placeholder Expressions}

For simple lambda expressions, you may use \textit{placeholders}:
\In{align*}{
    \texttt{?}{+}1 \mydefeq \Lambda_n n{+}1
}
Placeholder symbols can vary: $\texttt{?}$, $\texttt{-}$, $\texttt{1}$, etc.


\subsection{Family Declarations}

Syntax of the function application is world-standard and fixed:
\In{align*}{
    f(x) \text{ or } fx
}
, but sometimes you might want cuter syntax like that
\In{align*}{
    \strdgathered{\strdcellof{strd cellbase,sharp corners,regular polygon,regular polygon sides=5}{x}}
}
A \textit{family declaration} is an easy way to define arbitary application syntax:
\In{align*}{
    (\strdgathered{\strdcellof{strd cellbase,sharp corners,regular polygon,regular polygon sides=5}{x}} \in Y) _ {x \in X}
}
If you place a function implementation into it:
\In{align*}{
    (\strdgathered{\strdcellof{strd cellbase,sharp corners,regular polygon,regular polygon sides=5}{x}} \mydefeq x^2 \in Y) _ {x \in X}
}
, it might be called a \textit{family definition}.


\subsubsection{Dependent Functions}

Family declarations can do more. Let $F$ a set-valued function.
\In{align*}{
    (f (x) \in F(x)) _ {x \in X}
}
defines a function
\In{align*}{
    f : X \to \textstyle\bigcup _ {x \in X} F(x)
}
such that
\In{align*}{
    (\forall x \in X)(f(x) \in F(x))
}
Such $f$ is called a \textit{dependent function}, for the $F(x)$ depends on $x$. %
In case $F$ is a constant function,
\In{align*}{
    (f(x) \in Y)_{x \in X}
}
is a normal function $X \to Y$. 

\subsection{Russell's Paradox}
TODO

\subsection{Coherence}

You will write
\In{align*}{
    3+1+2
}
rather than
\In{align*}{
    (3+(0+1))+2
}
because you know the arithmetic laws
\In{align*}{
    x+(y+z) = (x+y)+z \\
    0+x = x = x+0
}
disambiguate unparenthesized expressions.
Informally a set of laws to introduce simpler syntax is called a \textit{coherence condition} or briefly \textit{coherence}.

