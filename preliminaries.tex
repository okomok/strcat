\section{Preliminaries}

% \begin{definition}[Vacuous Truth]
% Given boolean expressions $P$ and $Q$, a boolean expression
% \In{align*}{
%     P \implies Q
% } 
% is true if $P$ is false.
% As a consequence,
% \In{align*}{
%     (\forall x \in X)(Q(x)) \mydefeq (\forall x)(x \in X \implies Q(x))
% }
% is true in case $X$ is the empty set.
% \end{definition}


\subsection{Lambda Expressions}

\definitionof{Lambda Expression}{
    Following famous symbols like $\Sigma$, define a \emph{lambda expression} $\Lambda_x y$ as an anonymous function $x \mapsto y$.
}

\definitionof{Lambda-Tasted Form}{
    Given a function $\Gamma$ whose domain is a set of functions, %
    you can choose a short form of $\Gamma(\Lambda_x y)$ from the following \emph{lambda-tasted forms}
    \In{enumerate}{
        \item $\Gamma_x y$
        \item $\Gamma x.y$
        \item $(\Gamma x)(y)$
        \item $\Gamma x y$
    }
}


\definitionof{Placeholder Expression}{
    For simple lambda expressions, you may use \emph{placeholders} for example,
    \In{align*}{
        \texttt{?}{+}1 \mydefeq \Lambda_n n{+}1
    }
    Placeholder symbols can vary: $\texttt{?}$, $\texttt{-}$, $\texttt{1}$, etc.
}


\subsection{Universality}

\definitionof{Predicate}{
    We call a boolean-valued function a \emph{predicate}.
}

\definitionof{Universal Quantification}{
    Given a predicate $P$, we define a boolean value $\forall P$ as %
    ``given any $a$, it satisfies $P$".
}

\definitionof{Existential Quantification}{
    Given a predicate $P$, we define a boolean value $\exists P$ as %
    ``there exists some $a$ that satisfies $P$".
}

\In{definition}{
    Given a predicate $P$, another predicate $!P$ is defined as
    \In{align*}{
        !P(a) \mydefeq P(a) \wedge (\forall a\myprime)(P(a\myprime) \implies a = a\myprime)
    }
    using the third lambda-tasted form.
}

\definitionof{Uniqueness Quantification}{
    The \emph{uniqueness quantification} $\exists!$ is defined as $\exists \circ !$, %
    where $\circ$ is the function composition. %
    Spelling out the detail,
    \In{align*}{
        (\exists!a)(P(a)) = (\exists a)(!P(a))
    }
    meaning that ``there exists a unique $a$ that satisfies $P$".
}

\In{remark}{
    On the other hand,
    \In{math}{
        (\exists a)(!P(a) \wedge Q(a))
    }
    states ``there exists a unique $a$ that satisfies $P$. Furthermore, the $a$ satisfies $Q$".
}

\In{definition}{
    Given a prediate $P$ and a set $X$,
    \In{align*}{
        (\forall x \in X)(P(x)) &\mydefeq (\forall x)(x \in X \implies P(x)) \\
        (\exists x \in X)(P(x)) &\mydefeq (\exists x)(x \in X \wedge P(x))
    }
}

\definitionof{Universality}{
    Given a binary predicate $P$, we boldly call a statement of the form
    \In{align*}{
        (\forall x \in X)(\exists!y\in Y)(P(x,y))
    }
    the \emph{universality} of $P$.
}

\propositionof{Functional Universality}{
    Given a binary predicate $P$,
    \In{align*}{
        & (\forall x \in X)(\exists!y\in Y)(P(x,y)) \\
        \iff & (\exists f : X \to Y)(\forall x \in X)(\forall y \in Y)(P(x,y) \iff y = f(x))
    }
}

\In{strdproof}{
    $(\implies)$ by the Axiom of choice. $(\impliedby)$ immediate.
}

\definitionof{Functional Bijectivity}{
    Given a a function $g : Y \to X$, the statement
    \In{align*}{
        (\exists f : X \to Y)(\forall x \in X)(\forall y \in Y)(x = g(y) \iff y = f(x))
    }
    is known as the \emph{bijectivity} of $f$ and $g$. This is a special case of universality where $P(x,y)$ is $x = g(y)$.
}


\subsection{Families}

Syntax of function applications is world-standard:
\In{align*}{
    f(x)
}
but sometimes you might want cuter syntax like that
\newcommand*{\myxstar}{%
    \strdin{\strdpictureof{strd picture,baseline=(X.base)}}{%
        \strdcellof{strd cellbase,name=X,sharp corners,regular polygon,regular polygon sides=5}{x}
    }
}
\In{align*}{
    \myxstar
}

\definitionof{Family}{
    A \emph{family declaration} is a way to provide a function with arbitary application syntax. %
    The usage is clear from an example
    \In{align*}{
        (\myxstar \in Y) _ {x \in X}
    }
    We call it a \emph{family} of $Y$. Furthermore, a function body can be placed like that
    \In{align*}{
        (\myxstar \mydefeq x^2 \in Y) _ {x \in X}
    }
}

\In{example}{
    By definition any function $f : X \to Y$ is also a family $(f(x) \in Y)_{x \in X}$.
}

\In{example}{
    The most-used family declaration is the subscript style \In{math}{ (a_i)_i }. %
    You can view a tuple $(a_1,a_2,\ldots,a_n)$ to be an abbreviation of $(a_i)_{i\in \lbrace 1,2,\ldots,n\rbrace}$.
}

Families can do more. 
\definitionof{Dependent Function}{
    Let $F$ a set-valued function.
    \In{align*}{
        (f (x) \in F(x)) _ {x \in X}
    }
    defines a function
    \In{align*}{
        f : X \to \textstyle\bigcup _ {x \in X} F(x)
    }
    such that
    \In{align*}{
        (\forall x \in X)(f(x) \in F(x))
    }
    Such $f$ is called a \emph{dependent function}, for the $F(x)$ depends on $x$. %
    In case $F$ is a constant function, $f$ is a normal function $X \to Y$. 
}



\subsection{Coherence}

It is sure you write
\In{align*}{
    3+1+2
}
rather than
\In{align*}{
    (3+(0+1))+2
}
because you know the arithmetic laws
\In{align*}{
    x+(y+z) = (x+y)+z \\
    0+x = x = x+0
}
disambiguate unparenthesized expressions.
Informally laws to introduce natural syntax are called \emph{coherence conditions} or shortly \emph{coherence}.

