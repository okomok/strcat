
\section{Categories}

\subsection{The Definition}

\definitionof{Category}{
    A \emph{category} \cat{C} consists of
    \In{enumerate}{
        \item \emph{objects}: a class \catOb{C}
        \item \emph{morphisms} or \emph{hom-sets}: a family of sets
        \In{math}{
            (\cat{C}(A,B)) _ {A, B \in \catOb{C}}
        }
        \item \emph{compositions}: a family of functions
        \In{align*}{
            (\circ : \cat{C}(B,C) \times \cat{C}(A,B) \to \cat{C}(A,C) )_{A,B,C \in \catOb{C}}
        }
        \item \emph{identities} or \emph{units}: a family of morphisms
        \In{align*}{
            (\operatorname{id} _A \in \cat{C}(A,A))_{A \in \catOb{C}}
        }
    }
    satisfying the following coherence conditions
    \In{enumerate}{
        \item \emph{associativity}: for any $f \in \cat{C}(A,B)$, $g \in \cat{C}(B,C)$, and $h \in \cat{C}(C,D)$,
        \In{align*}{
            h \circ (g \circ f) = (h \circ g) \circ f
        }
        \item \emph{unitality}: for any $f \in \cat{C}(A,B)$,
        \In{align*}{
            \operatorname{id}_B \circ f = f = f \circ \operatorname{id}_A
        }
    }
    A morphism $f \in \cat{C}(A,B)$ is often denoted as
    \In{math}{
        f : A \to B
    }.
}

\subsection{String Diagrams}

From now on, we will introduce \emph{string diagrams} %
to complement(or hopefully replace) commutative diagrams.
\mynewline
\indent Given a category \cat C, an object $A$ is depicted as an optionally-tagged string
\In{align*}{
    \strdgathered{
        \strdmid,\strdrlabel{A}
        ,\strdstartlabel{\cat{C}}
    }
}
A morphism $f : A \to B$ is depicted as a node
\In{align*}{
    \strdgathered{
        \strdmid,\strdrlabel{B}
        ,\strdcell{f}
        ,\strdmid,\strdrlabel{A}
    }
}
The composition joins two strings:
\In{align*}{
    \strdgathered{
        \strdmid,\strdrlabel{C}
        ,\strdcell{g}
        ,\strdmid,\strdrlabel{B}
        ,\strdcell{f}
        ,\strdmid,\strdrlabel{A}
    } \,\mydefeq g \circ f
}
Identity morphisms are indistinguishable from objects:
\In{align*}{
    \strdgathered{
        \strdmid,\strdrlabel{A}
    } \,\mydefeq \operatorname{id}_A
}
Check these diagrams create no ambiguity thanks to the coherence.


\definitionof{Isomorphism}{
    Given objects $A$ and $B$, we call a pair of morphisms
    \In{align*}{
        f &: A \to B\\
        g &: B \to A
    }
    an \emph{isomorphism} or shortly \emph{iso} %
    provided that the following \emph{invertibility} is satisfied.
    \In{align*}{
        \strdgathered{
            \strdmid,\strdrlabel{A}
            ,\strdcell{g}
            ,\strdmid,\strdrlabel{B}
            ,\strdcell{f}
            ,\strdmid,\strdrlabel{A}
        } =\
        \strdgathered{
            \strdmid,\strdrlabel{A}
        } \ \ \text{and}\ \
        \strdgathered{
            \strdmid,\strdrlabel{B}
            ,\strdcell{f}
            ,\strdmid,\strdrlabel{A}
            ,\strdcell{g}
            ,\strdmid,\strdrlabel{B}
        } =\
        \strdgathered{
            \strdmid,\strdrlabel{B}
        }
    }
    Each morphism of the pair is also called an \emph{isomorphism}.
}

\definitionof{Functional Box}{
    Given categories $\cat{C}$ and $\cat{C}\myprime$, a function
    \In{align*}{
        h : \cat{C}(A,B) \to \cat{C}\myprime(A\myprime,B\myprime)
    }
    is depicted as a \emph{functional box}
    \In{align*}{
        \strdgathered{
            \strdmid,\strdrlabel{B\myprime}
            ,\strdbox{1}{1}{
                \strdmid,\strdrlabel{B}
                ,\strdcell{?}
                ,\strdmid,\strdrlabel{A}
            },\strdat{Left}{\strdcellof{strd overlabel,font=\normalsize}{h}}
            ,\strdmid,\strdrlabel{A\myprime}
        }
    }
}

\definitionof{Opposite Category}{
    Given a category $\cat{C}$ and a morphism
    \In{align*}{
        \strdgathered{
            \strdmid,\strdrlabel{B}
            ,\strdcell{f}
            ,\strdmid,\strdrlabel{A}
        }
    }
    you can build a category with strings upside down:
    \In{align*}{
        \strdgathered{
            \strdmid,\strdrlabel{A}
            ,\strdcell{\strdupsidedown{f}}
            ,\strdmid,\strdrlabel{B}
        }
    }
    which is denoted as $\cat{C}^{\operatorname{op}}$, the \emph{opposite category} of $\cat{C}$.
}

\definitionof{Discrete Category}{
    A category $\cat{C}$ such that
    \In{align*}{
        A = B &\implies \cat{C}(A,B) = \lbrace \operatorname{id}_A \rbrace\\
        A \neq B &\implies \cat{C}(A,B) = \emptyset
    }
    is called a \emph{discrete category}.
    Any set can be represented as a discrete category.
}

\definitionof{Product Category}{
    Given two categories $\cat{A}$ and $\cat{B}$, the \emph{product category}
    \In{align*}{
        \cat{A} \times \cat{B}
    }
    is depicted as parallel strings
    \In{align*}{
        \strdgathered{
            \strdfork{
                \strdxscale{1.5}\strdljump
                ,\strdmid,\strdrlabel{A\myprime},\strdstartlabel{\cat{A}}
                ,\strdcell{a}
                ,\strdmid,\strdrlabel{A}
            }
            ,\strdfork{
                \strdmid,\strdrlabel{B\myprime},\strdstartlabel{\cat{B}}
                ,\strdcell{b}
                ,\strdmid,\strdrlabel{B}
            }
        }
    }
}
A composition, which joins parallel strings, is defined by
\In{align*}{
    \strdgathered{
        \strdfork{
            \strdxscale{0.7}\strdljump
            ,\strdmid
            ,\strdcell{a\myprime}
            ,\strdmid
            ,\strdmid
            ,\strdcell{a}
            ,\strdmid
        }
        ,\strdfork{
            \strdxscale{0.7}\strdrjump
            ,\strdmid
            ,\strdcell{b\myprime}
            ,\strdmid
            ,\strdmid
            ,\strdcell{b}
            ,\strdmid
        }
        ,\strdfork{
            \strdyscale{2}\strdjump
            ,\strdxscale{2.7}{\strdcutoff\strdminus}
        }
    } \mydefeq 
    \strdgathered{
        \strdfork{
            \strdxscale{0.7}\strdljump
            ,\strdmid
            ,\strdcell{a\myprime}
            ,\strdmid
            ,\strdmid
            ,\strdcell{a}
            ,\strdmid
        }
        ,\strdfork{
            \strdxscale{0.7}\strdrjump
            ,\strdmid
            ,\strdcell{b\myprime}
            ,\strdmid
            ,\strdmid
            ,\strdcell{b}
            ,\strdmid
        }
        ,\strdfork{
            \strdyscale{4}{\strdcutoff\strdmid}
        }
    }
}
An identity is trivially
\In{align*}{
    \strdgathered{
        \strdfork{
            \strdxscale{0.7}\strdljump
            ,\strdmid
        }
        ,\strdfork{
            \strdxscale{0.7}\strdrjump
            ,\strdmid
        }
    }
}
By these definitions,
\In{align*}{
    \strdgathered{
        \strdfork{
            \strdxscale{0.7}\strdljump
            ,\strdmid
            ,\strdmid
            ,\strdcell{a}
            ,\strdmid
        }
        ,\strdfork{
            \strdxscale{0.7}\strdrjump
            ,\strdmid
            ,\strdcell{b}
            ,\strdmid
            ,\strdmid
        }
    } \,=\,
    \strdgathered{
        \strdfork{
            \strdxscale{0.7}\strdljump
            ,\strdmid
            ,\strdcell{a}
            ,\strdmid
            ,\strdmid
        }
        ,\strdfork{
            \strdxscale{0.7}\strdrjump
            ,\strdmid
            ,\strdmid
            ,\strdcell{b}
            ,\strdmid
        }
    }
}

