
\section{Categories}

\subsection{Definition}

A \textit{category} $\mathcal{C}$ consists of:
\In{enumerate}{
\item \textit{Objects}: a class
    \In{align*}{
        \operatorname{Ob}\mathcal{C}
    }
\item \textit{Morphisms}: a family of sets
    \In{align*}{
        (\mathcal{C}(A,B)) _ {A, B \in \operatorname{Ob}\mathcal{C}}
    }
\item \textit{Compositions}: a family of functions
    \In{align*}{
        (\circ : \mathcal{C}(B,C) \times \mathcal{C}(A,B) \to \mathcal{C}(A,C) )_{A,B,C \in \operatorname{Ob}\mathcal{C}}
    }
\item \textit{Idenitities} or \textit{units}: a family of morphisms
    \In{align*}{
        (\operatorname{id} _A \in \mathcal{C}(A,A))_{A \in \operatorname{Ob}\mathcal{C}}
    }
}
satisfying the following coherence conditions:
\In{enumerate}{
\item \textit{Associativity}: for any $f \in \mathcal{C}(A,B)$, $g \in \mathcal{C}(B,C)$, and $h \in \mathcal{C}(C,D)$,
    \In{align*}{
        h \circ (g \circ f) = (h \circ g) \circ f
    }
\item \textit{Unitality}: for any $f \in \mathcal{C}(A,B)$,
    \In{align*}{
        \operatorname{id}_B \circ f = f = f \circ \operatorname{id}_A
    }
}
A morphism $f \in \mathcal{C}(A,B)$ will often be denoted by $f : A \to B$.

\subsection{String Diagrams}

In this document, \textit{commutative diagrams} are never used. Instead, %
we represent an object $A$ by a tagged string
\In{align*}{
    \strdgathered{
        \strdmid,\strdrlabel{A}
        ,\strdstartlabel{\mathcal{C}}
    }
}
and a morphism $f \in \mathcal{C}(A,B)$ by a node
\In{align*}{
    \strdgathered{
        \strdmid,\strdrlabel{B}
        ,\strdcell{f}
        ,\strdmid,\strdrlabel{A}
    }
}
A composition joins two strings:
\In{align*}{
    \strdgathered{
        \strdmid,\strdrlabel{C}
        ,\strdcell{g}
        ,\strdmid,\strdrlabel{B}
        ,\strdcell{f}
        ,\strdmid,\strdrlabel{A}
    } \,\mydefeq g \circ f
}
An idenity is indistinguishable from an object:
\In{align*}{
    \strdgathered{
        \strdmid,\strdrlabel{A}
    } \,\mydefeq \operatorname{id}_A
}
Check the string notations create no ambiguity thanks to the coherence.

\subsection{Functional Boxes}

Given categories $\mathcal{C}$ and $\mathcal{C}'$, a function
\In{align*}{
    h : \mathcal{C}(A,B) \to \mathcal{C}'(A',B')
}
is denoted by
\In{align*}{
    \strdgathered{
        \strdmid,\strdrlabel{B'}
        ,\strdbox{1}{1}{
            \strdmid,\strdrlabel{B}
            ,\strdcell{?}
            ,\strdmid,\strdrlabel{A}
        },\strdat{Left}{\strdcellof{strd overlabel,font=\normalsize}{h}}
        ,\strdmid,\strdrlabel{A'}
    }
}


\subsection{Opposite Categories}

Given a category $\mathcal{C}$ and a morphism
\In{align*}{
    \strdgathered{
        \strdmid,\strdrlabel{B}
        ,\strdcell{f}
        ,\strdmid,\strdrlabel{A}
    }
}
You can build a category with strings upsidedown:
\In{align*}{
    \strdgathered{
        \strdmid,\strdrlabel{A}
        ,\strdcell{\strdupsidedown{f}}
        ,\strdmid,\strdrlabel{B}
    }
}
, which is denoted by $\mathcal{C}^{\operatorname{op}}$, the \textit{opposite category} of $\mathcal{C}$.

\subsection{Discrete Categories}

A category $\mathcal{C}$ such that
\In{align*}{
    A \neq B \implies \mathcal{C}(A,B) = \emptyset
}
is called a \textit{discrete} category.
A set is equivalent to a discrete category.


\subsection{Product Categories}
TODO 

