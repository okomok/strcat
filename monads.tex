\section{Monads}

\subsection{The Definition}

\definitionof{Monad}{
    Given a category \cat C, a \emph{monad} on \cat C consists of
    \In{enumerate}{
        \item a functor $T : \mathcal{C} \to \mathcal{C}$
        \item \emph{unit}: a natural transformation $\eta : \operatorname{Id}_T \to T$
        \item \emph{multiplication}: a natural transformation $\mu : T \circ T \to T$
    }
    satisfying the coherence conditions
    \In{enumerate}{
        \item \emph{associativity}: %
        \In{math}{
            \mu \circ T\mu = \mu \circ \mu T
        }
        \item \emph{unitality}: %
        \In{math}{
            \mu \circ T\eta = \operatorname{Id}_T = \mu \circ \eta T 
        }
    }
}

A unit and multiplication are depicted respectively as
\In{align*}{
    \strdgathered{
        \strdfork{ \strdmid,\strdmid }
        ,\strdtubestartline
        ,\strdtube\strdtubemid
        ,\strdtwincell{\eta}
        %,\strdnone{\strdtubesqcap{1}{1}}
    }\ \
    \strdgathered{
        \strdfork{ \strdmid,\strdmid }
        ,\strdtubestartline
        ,\strdtube{
            \strdtubemid
            ,\strdtwincell{\mu}
            ,\strdfork{
                \strdtubemid
            }
            ,\strdtubecap{1}{1}
            ,\strdtubeendline
        }
    }
}
The associativity is depicted as
\In{align*}{
    \strdgathered{
        \strdfork{\strdmid,\strdmid,\strdmid}
        ,\strdtubestartline
        ,\strdtube{
            \strdtubemid
            ,\strdtwincell{\mu}
            ,\strdfork{
                \strdtubemid
                ,\strdtwincell{\mu}
                ,\strdfork{
                    \strdtubemid
                }
                ,\strdtubecap{1}{1}
            }
            ,\strdscale{2}{2}{\strdtubecap{1}{1}}
            ,\strdtubeendline
        }
    } =
    \strdgathered{
        \strdfork{\strdmid,\strdmid,\strdmid}
        ,\strdtubestartline
        ,\strdtube{
            \strdtubemid
            ,\strdtwincell{\mu}
            ,\strdfork{
                \strdtubemid,\strdtubemid
            }
            ,\strdtubecap{1}{1}
            ,\strdtwincell{\mu}
            ,\strdfork{
                \strdtubemid
            }
            ,\strdtubecap{1}{1}
            ,\strdtubeendline
        }
    }
}
This inspires you to assign
\In{align*}{
    \strdgathered{
        \strdfork{\strdmid,\strdmid}
        ,\strdtubestartline
        ,\strdtube{
            \strdtubemid
            ,\strdtwincell{\mu}
            ,\strdfork{
                \strdtubemid
            }
            ,\strdfork{
                \strdtubecap{1}{1}
            }
            ,\strdfork{
                \strdtubecap{2}{2}
                ,\strdtubeendline
            }
        }
    }
}
The unitality is
\In{align*}{
    \strdgathered{
        \strdfork{\strdmid,\strdmid,\strdmid}
        ,\strdtubestartline
        ,\strdtube{
            \strdtubemid
            ,\strdtwincell{\mu}
            ,\strdfork{
                \strdtubemid
                ,\strdtwincell{\eta}
            }
            ,\strdtubecap{1}{1}
            ,\strdtubemid
            ,\strdnone{\strdtwincell{\eta}}
            ,\strdtubeendline
        }
    } =
    \strdgathered{
        \strdfork{\strdmid}
        ,\strdtubestartline
        ,\strdtube{
            \strdtubemid
            ,\strdtubeendline
        }
    } =
    \strdgathered{
        \strdfork{\strdmid,\strdmid,\strdmid}
        ,\strdtubestartline
        ,\strdtube{
            \strdtubemid
            ,\strdtwincell{\mu}
            ,\strdfork{
                \strdtubemid
                ,\strdtubemid
                ,\strdtubeendline
            }
            ,\strdtubecap{1}{1}
            ,\strdtwincell{\eta}
        }
    }
}


\definitionof{Monad Morphism}{
    Given a category $\cat C$, a \emph{monad morphism} consists of
    \In{enumerate}{
        \item \emph{domain}: a monad $(T, \eta, \mu)$ on \cat C
        \item \emph{codomain}: a monad $(T\myprime, \eta\myprime, \mu\myprime)$ on \cat C 
        \item a natural transformation $\tau : T \to T\myprime$
    }
    satisfying the coherence conditions
    \In{enumerate}{
        \item \emph{multiplication-compatibility}:
        \In{math}{
            \tau \circ \eta = \eta\myprime
        }
        \item \emph{unit-compatibility}:
        \In{math}{
            \tau \circ \mu = \mu\myprime \circ \tau\tau
        }
    }
}

The coherence is depicated as
\In{align*}{
    \strdgathered{
        \strdfork{\strdmid,\strdmid,\strdmid}
        ,\strdtubestartline
        ,\strdtube{
            \strdtubemid
            ,\strdtwincell{\mu\strdrlap\myprime}
            ,\strdfork{
                \strdtubemid
                ,\strdtwincell{\tau}
                ,\strdtubemid
            }
            ,\strdtubecap{1}{1}
            ,\strdtwincell{\tau}
            ,\strdtubemid
            ,\strdtubeendline
        }
    } &=
    \strdgathered{
        \strdfork{\strdmid,\strdmid,\strdmid}
        ,\strdtubestartline
        ,\strdtube{
            \strdtubemid
            ,\strdtwincell{\tau}
            ,\strdtubemid
            ,\strdtwincell{\mu}
            ,\strdfork{
                \strdtubemid
            }
            ,\strdtubecap{1}{1}
            ,\strdtubeendline
        }
    } \\
    \strdgathered{
        \strdfork{\strdmid,\strdmid,\strdmid}
        ,\strdtubestartline
        ,\strdtube{
            \strdtubemid
            ,\strdtwincell{\tau}
            ,\strdtubemid
            ,\strdtwincell{\eta}
        }
    } &=
    \strdgathered{
        \strdfork{\strdmid,\strdmid}
        ,\strdtubestartline
        ,\strdtube{
            \strdtubemid
            ,\strdtwincell{\eta\strdrlap\myprime}
        }
    } 
}

\definitionof{Categoy of Monads}{
    Given a category \cat C, the \emph{category of monads} \catMnd C is a category whose objects are %
    monads and whose morphisms are monad morphisms.
}

\subsection{Kleisli Categories}

\definitionof{Kleisli Category}{
    Given a monad $(T,\eta,\mu)$ on \cat C, the \emph{Kleisli category} of $T$, %
    denoted as $\cat C_T$ is a category consisting of
    \In{enumerate}{
        \item $\text{Ob}(\cat C_T) \mydefeq \catOb C$
        \item $\cat C_T(A,B) \mydefeq \cat C(A, TB)$
        \item $l \circ k \mydefeq \mu \circ T(l) \circ k$
        \item $\operatorname{id}_A \mydefeq \eta_A$
    }
}

In diagrams, a morphism in $\cat C_T$ is depicted as a \emph{Kleisli box}
\In{align*}{
    \strdgathered{
        \strdmid,\strdrlabel{B},\strdstartlabel{\cat C_T}
        ,\strdbox{1.7}{1.7}{
            \strdfork{\strdmid}
            ,\strdtubestart
            ,\strdtube\strdtubemid
            ,\strdwidecell{1}{1}{k}
            ,\strdmid
        }
        ,\strdmid,\strdrlabel{A}
    }
}
The composition is defined as
\In{align*}{
    \strdgathered{
        \strdmid,\strdrlabel{C}
        ,\strdbox{1.7}{1.7}{
            \strdfork{\strdmid}
            ,\strdtubestart
            ,\strdtube\strdtubemid
            ,\strdwidecell{1}{1}{l}
            ,\strdmid
        }
        ,\strdmid,\strdrlabel{B}
        ,\strdbox{1.7}{1.7}{
            \strdfork{\strdmid}
            ,\strdtubestart
            ,\strdtube\strdtubemid
            ,\strdwidecell{1}{1}{k}
            ,\strdmid
        }
        ,\strdmid,\strdrlabel{A}
    } \mydefeq
    \strdgathered{
        \strdmid,\strdrlabel{C}
        ,\strdbox{2.5}{2.5}{
            \strdfork{
                \strdmid,\strdmid,\strdmid,\strdrlabel{B}
            }
            ,\strdtubestart
            ,\strdtube{
                \strdtubemid
                ,\strdtwincell{\mu}
                ,\strdfork{
                    \strdtubemid
                    ,\strdwidecell{1}{1}{l}
                }
                ,\strdtubecap{1}{1}
                ,\strdtubemid
                ,\strdwidecell{2}{2}{k}
            }
            ,\strdmid
        }
        ,\strdmid,\strdrlabel{A}
    }
}
An identity morphism is defined as
\In{align*}{
    \strdgathered{
        \strdmid,\strdrlabel{A}
        ,\strdbox{1.7}{1.7}{
            \strdfork{\strdmid}
            ,\strdtubestart
            ,\strdtube\strdtubemid
            ,\strdtwincell{\eta}
            ,\strdmid
        }
        ,\strdmid,\strdrlabel{A}
    }
}



\definitionof{Kleisli Adjunction}{
    Define a functor $L : \cat C \to \cat C_T$ as
    \In{align*}{
        \strdgathered{
            \strdmid,\strdrlabel{}
            ,\strdbox{1.7}{1.7}{
                \strdfork{\strdmid,\strdcell{\catph{?}}}
                ,\strdtubestart
                ,\strdtube\strdtubemid
                ,\strdtwincell{\eta}
                ,\strdmid
            }
            ,\strdmid,\strdrlabel{}
        }
    }
    $K : \cat C_T \to \cat C$ as
    \In{align*}{
        \strdgathered{
            \strdmid,\strdrlabel{}
            ,\strdbox{1.7}{1.7}{
                \strdfork{\strdmid}
                ,\strdtubestart
                ,\strdtube\strdtubemid
                ,\strdwidecell{1}{1}{k}
                ,\strdmid
            }
            ,\strdmid,\strdrlabel{}
        } \mapsto
        \strdgathered{
            \strdfork{
                \strdmid,\strdmid,\strdmid
            }
            ,\strdtubestartline
            ,\strdtube{
                \strdtubemid
                ,\strdtwincell{\mu}
                ,\strdfork{
                    \strdtubemid
                    ,\strdwidecell{1}{1}{k}
                }
                ,\strdtubecap{1}{1}
                ,\strdtubemid
                ,\strdtubeendline
            }
        }
    }
    then they consitute the \emph{Kleisli adjunction} $L \dashv K$ whose adjunct is %
    the Kleisli boxing.
}

