\section{The Yoneda Lemma}


\begin{definition}
Given a functor $F : \cat{C}\catop \to \catSet$ and an object $A$ in $\cat{C}$, %
a natural tranformation of the form
\In{align*}{
    (\tau_X &: \cat{C}(X,A) \to FX)_ X
}
can be depicted as
\In{align*}{
    \strdgathered{
        \strdfork{
            \strdmid
            ,\strdcell{\strdupsidedown{\catph{?}}}
            ,\strdmid,\strdrlabel{A}
        }
        ,\strdtubestart
        ,\strdcutoff\strdtubebound
        ,\strdtube{
            \strdtubemid,\strdat{Left}{\strdoverlabel{F}}
        }
        ,\strdtwincell{\tau}
        ,\strdtubeof{strd frame}\strdtubeclose
    } \mydefeq
    \strdgathered{
        \strdfork{
            \strdmid
        }
        ,\strdtubestart
        ,\strdcutoff\strdtubebound
        ,\strdtube{
            \strdtubemid,\strdat{Left}{\strdoverlabel{F}}
        }
        ,\strdbox{1}{1}{
            \strdmid
            ,\strdcell{\strdupsidedown{\catph{?}}}
            ,\strdmid,\strdrlabel{A}
        },\strdat{Left}{\strdcellof{strd overlabel,font=\normalsize}{\tau}}
    }
}
owing to the naturality.
\end{definition}

\begin{definition}[Yoneda Bijection]
\label{y.b}
The \emph{Yoneda bijection} is defined as 
\In{align*}{
    \mylhs{\catNat_X(\cat{C}(X,A),FX)} &\cong \myrhs{FA}\\
    \strdgathered{
        \strdfork{
            \strdmid
            ,\strdcell{\strdupsidedown{\catph{?}}}
            ,\strdmid,\strdrlabel{A}
        }
        ,\strdtubestart
        ,\strdcutoff\strdtubebound
        ,\strdtube{
            \strdtubemid%,\strdat{Left}{\strdoverlabel{F}}
        }
        ,\strdtwincell{\tau}
        ,\strdtubeof{strd frame}\strdtubeclose
    } &\mapsto
    \strdgathered{
        \strdfork{
            \strdmid
            ,\strdmid,\strdrlabel{A}
        }
        ,\strdtubestart
        ,\strdcutoff\strdtubebound
        ,\strdtube{
            \strdtubemid%,\strdat{Left}{\strdoverlabel{F}}
        }
        ,\strdtwincell{\tau}
        ,\strdtubeof{strd frame}\strdtubeclose
    }\\
    \strdgathered{
        \strdfork{
            \strdmid
            ,\strdcell{\strdupsidedown{\catph{?}}}
            ,\strdmid,\strdrlabel{A}
        }
        ,\strdtube{
            \strdtubestartline
            ,\strdtubemid
            ,\strdnone{\strdtwincell{\tau}}
            ,\strdtubemid
            ,\strdwidecell{1}{1}{a}
        }
    } &\mymapsfrom
    \strdgathered{
        \strdfork{
            \strdmid,\strdrlabel{A}
        }
        ,\strdtube{
            \strdtubestartline
            ,\strdtubemid
            ,\strdnone{\strdtwincell{\tau}}
            ,\strdwidecell{1}{1}{a}
        }
    }
}
\end{definition}

\begin{lemma}[Yoneda Lemma]
\label{y.l}
The Yoneda bijection is actually bijective and natural in $F$ and $A$. 
\end{lemma}
\begin{strdproof}
Now the proof is on my soul trivial!
\end{strdproof}


\begin{definition}[Yoneda Embedding]
The \emph{Yoneda embedding} is defined as
\In{align*}{
    \mylhs{\Lambda_A \Lambda_X \cat{C}(X,A) : \cat{C}} &\to \myrhs{[\cat{C}\catop,\catSet]}\\
    \strdgathered{
        \strdmid,\strdrlabel{B}
        ,\strdcell{f}
        ,\strdmid,\strdrlabel{A}
    } &\mapsto\,
    \strdgathered{
        \strdtubeof{}{
            \strdtubestart%line
            ,\strdtubemid,\strdbackground{strd roadcolor}
            ,\strdat{Left}{\strdrlabel{B}}
            ,\strdat{Left End}{\strdcell{\Identity{f}}}
            ,\strdtubemid,\strdbackground{strd roadcolor}
            ,\strdat{Left}{\strdrlabel{A}}
        }
    } \,=\,
    \strdgathered{
        \strdmid,\strdrlabel{B}
        ,\strdcell{f}
        ,\strdmid,\strdrlabel{A}
        ,\strdcell{\catph{?}}
        ,\strdmid
    } 
}
using the diagram of hom functors. In short,
\In{align*}{
    \strdgathered{
        \strdtubeof{}{
            \strdtubestart%line
            ,\strdtubemid,\strdbackground{strd roadcolor}
            ,\strdat{Left End}{\strdcell{\texttt{?}}}
            ,\strdtubemid,\strdbackground{strd roadcolor}
        }
    }
}
\end{definition}

\begin{definition}
A natural transformation of the form
\In{align*}{
    (\tau_X : \cat{C}(X,A) \to \cat{C}(X,B))_X
}
can be depicted as
\In{align*}{
    \strdgathered{
        \strdmid,\strdrlabel{B}
        ,\strdfork{
            \strdmid,\strdrlabel{A}
            ,\strdcell{\catph{?}}
            ,\strdmid
        }
        ,\strdtubeof{strd frame}{
            \strdtubestart
            ,\strdtubeunclose
            ,\strdtwincell{\tau}
        }
    }
}
\end{definition}

\begin{definition}[Yoneda Embedding Bijection]
In special case $F \mydefeq \cat{C}(\catph{-},B)$, the Yoneda bijection is expanded to
\In{align*}{
    \mylhs{\catNat_X(\cat{C}(X,A),\cat{C}(X,B))} &\cong \myrhs{\cat{C}(A,B)}\\
    \strdgathered{
        \strdmid,\strdrlabel{B}
        ,\strdfork{
            \strdmid,\strdrlabel{A}
            ,\strdcell{\catph{?}}
            ,\strdmid
        }
        ,\strdtubeof{strd frame}{
            \strdtubestart
            ,\strdtubeunclose
            ,\strdtwincell{\tau}
        }
    } &\mapsto
    \strdgathered{
        \strdmid,\strdrlabel{B}
        ,\strdfork{
            \strdmid,\strdrlabel{A}
            ,\strdmid
        }
        ,\strdtubeof{strd frame}{
            \strdtubestart
            ,\strdtubeunclose
            ,\strdtwincell{\tau}
        }
    } \\
    \strdgathered{
        \strdnone{
            \strdfork{
                \strdtubestart
                ,\strdtubemid
                ,\strdtwincell{\tau}
            }
        }
        ,\strdmid,\strdrlabel{B}
        ,\strdcell{f}
        ,\strdmid,\strdrlabel{A}
        ,\strdcell{\catph{?}}
        ,\strdmid
    } & \mymapsfrom
    \strdgathered{
        \strdnone{
            \strdfork{
                \strdtubestart
                ,\strdtubemid
                ,\strdtwincell{\tau}
            }
        }
        ,\strdmid,\strdrlabel{B}
        ,\strdcell{f}
        ,\strdmid,\strdrlabel{A}
    }
}
\end{definition}
You will notice the second mapping is the Yoneda embedding so that %
it is full and faithful. Combined with \cref{ff.pr.inv},
\begin{proposition}[Yoneda Principle]
\label{y.p}
\In{align*}{
    \strdgathered{
        \strdmid,\strdrlabel{B}
        ,\strdfork{
            \strdmid,\strdrlabel{A}
            ,\strdcell{\catph{?}}
            ,\strdmid
        }
        ,\strdtubeof{strd frame}{
            \strdtubestart
            ,\strdtubeunclose
            ,\strdtwincell{\tau}
        }
    } : \text{invertible} \iff
    \strdgathered{
        \strdmid,\strdrlabel{B}
        ,\strdfork{
            \strdmid,\strdrlabel{A}
            ,\strdmid
        }
        ,\strdtubeof{strd frame}{
            \strdtubestart
            ,\strdtubeunclose
            ,\strdtwincell{\tau}
        }
    } : \text{invertible}
}
\end{proposition}

