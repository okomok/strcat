\section{Natural Transformations}

\subsection{Definition}

Given two infrafunctors
\In{align*}{
    (F(f) \in \mathcal{D}(FA,FB)) _ {f \in \mathcal{C}(A,B)}\\
    (G(f) \in \mathcal{D}(GA,GB)) _ {f \in \mathcal{C}(A,B)}
}
, a family of morphisms
\In{align*}{
    (\tau_A \in \catMor{D}(FA,GA)) _ {A \in \catOb{C}}
}
is \textit{natural} if for any $f \in \mathcal{C}(A,B)$,
\In{align*}{
    \tau_B \circ F(f) = G(f) \circ \tau_A
}
In particular case $F$ and $G$ are functorial(then they are functors), $\tau$ is called a \textit{natural transformation}:
\In{align*}{
    \tau : F \to G
}

\subsection{Natural Connectors}

In string diagrams, a natural transformation is a connector of two functors:
\In{align*}{
    \strdgathered{
        \strdfork{
            \strdmid,\strdrlabel{}
            ,\strdmid,\strdrlabel{}
        }
        ,\strdtube{
            \strdtubestartline
            ,\strdtubemid,\strdat{Left}{\strdoverlabel{G}}
            ,\strdtwincell{\tau}
            ,\strdtubemid,\strdat{Left}{\strdoverlabel{F}}
            ,\strdtubeendline
        }
    }
}
, for the naturality guarantees
\In{align*}{
    \strdgathered{
        \strdfork{
            \strdmid,\strdrlabel{B}
            ,\strdmid,\strdrlabel{}
            ,\strdcell{f}
            ,\strdmid,\strdrlabel{A}
        }
        ,\strdtube{
            \strdtubestartline
            ,\strdtubemid
            ,\strdtwincell{\tau}
            ,\strdtubemid
            ,\strdtubemid
            ,\strdtubeendline
        }
    } =
    \strdgathered{
        \strdfork{
            \strdmid,\strdrlabel{B}
            ,\strdcell{f}
            ,\strdmid,\strdrlabel{}
            ,\strdmid,\strdrlabel{A}
        }
        ,\strdtube{
            \strdtubestartline
            ,\strdtubemid
            ,\strdtubemid
            ,\strdtwincell{\tau}
            ,\strdtubemid
            ,\strdtubeendline
        }
    }
}
, which inspires you to define them as
\In{align*}{
    \strdgathered{
        \strdfork{
            \strdmid,\strdrlabel{B}
            ,\strdcell{f}
            ,\strdmid,\strdrlabel{A}
        }
        ,\strdtube{
            \strdtubestartline
            ,\strdtubemid%,\strdat{Left}{\strdoverlabel{G}}
            ,\strdtwincell{\tau}
            ,\strdtubemid%,\strdat{Left}{\strdoverlabel{F}}
            ,\strdtubeendline
        }
    }
}



\subsection{Vertical Composition}

Given three \textit{parallel} functors
\In{align*}{
    F : \mathcal{C} \to \mathcal{D}\\
    G : \mathcal{C} \to \mathcal{D}\\
    H : \mathcal{C} \to \mathcal{D}
}
and two natural transformations
\In{align*}{
    \tau &: F \to G\\
    \sigma &: G \to H
}
, the \textit{vertical composition} of $\tau$ and $\sigma$ is defined by
\In{align*}{
    \strdgathered{
        \strdfork{
            \strdmid,\strdrlabel{}
            ,\strdmid,\strdrlabel{}
            ,\strdmid,\strdrlabel{}
        }
        ,\strdtube{
            \strdtubestartline
            ,\strdtubemid,\strdat{Left}{\strdoverlabel{H}}
            ,\strdtwincell{\sigma}
            ,\strdtubemid,\strdat{Left}{\strdoverlabel{G}}
            ,\strdtwincell{\tau}
            ,\strdtubemid,\strdat{Left}{\strdoverlabel{F}}
            ,\strdtubeendline
        }
    }
}
The naturality is trivial; You can travel by bus and train.

\subsection{Horizontal Composition}

Given four functors
\In{align*}{
    F &: \mathcal{A} \to \mathcal{B}\\
    G &: \mathcal{A} \to \mathcal{B}\\
    H &: \mathcal{B} \to \mathcal{C}\\
    K &: \mathcal{B} \to \mathcal{C}
}
and two natural transformations
\In{align*}{
    \tau &: F \to G\\
    \sigma &: H \to K
}
, the \textit{horizontal composition} of $\tau$ and $\sigma$ is defined by
\In{align*}{
    \strdgathered{
        \strdfork{
            \strdmid,\strdrlabel{}
            ,\strdmid,\strdrlabel{}
        }
        ,\strdtube{
            \strdtubestart
            ,\strdfork{
                \strdtubemid,\strdat{Left}{\strdoverlabel{G}}
                ,\strdtwincell{\tau}
                ,\strdtubemid,\strdat{Left}{\strdoverlabel{F}}
            }
            ,\strdtubeextend{1}{1}
            ,\strdcutoff\strdtubebound
            ,\strdtubemid,\strdat{Left}{\strdoverlabel{K}}
            ,\strdtwincell{\sigma}
            ,\strdtubemid,\strdat{Left}{\strdoverlabel{H}}
            ,\strdcutoff\strdtubebound
        }
    }
}
You can easily check the naturality; Travel by car ferry.


\subsection{Identity Natural Transformations}

Given a functor
\In{align*}{
    F &: \mathcal{C} \to \mathcal{D}
}
, the \textit{identity natural transformation}
\In{align*}{
    i _F : F \to F
}
is defined by
\In{align*}{
    \strdgathered{
        \strdfork{
            \strdmid,\strdrlabel{}
        }
        ,\strdtube{
            \strdtubestartline
            ,\strdtubemid
            ,\strdtubeat{Mid}{\strdtwincell{i}}
            ,\strdtubeendline
        }
    } \mydefeq\,
    \strdgathered{
        \strdfork{
            \strdmid,\strdrlabel{}
        }
        ,\strdtube{
            \strdtubestartline
            ,\strdtubemid
            ,\strdtubeendline
        }
    }
}

