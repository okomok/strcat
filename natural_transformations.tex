\section{Natural Transformations}

\subsection{The Definition}

\definitionof{Naturality}{
    Given two infrafunctors
    \In{align*}{
        F, G : \cat C \to \cat D
    }
    a family of morphisms
    \In{align*}{
        (\tau_A \in \cat{D}(FA,GA)) _ {A \in \catOb{C}}
    }
    is called \emph{natural} when for any $f \in \cat{C}(A,B)$,
    \In{align*}{
        \tau_B \circ F(f) = G(f) \circ \tau_A
    }
    In case parenthese are cumbersome, you can say ``$\tau_A$ is \emph{natural in} $A$".
}

\definitionof{Natural Transformation}{
    Furthermore, in particular case $F$ and $G$ are functorial(then they are functors), %
    $\tau$ is denoted as a \emph{natural transformation}
    \In{align*}{
        \tau : F \to G
    }
}

\In{remark}{
    Naturality is defined to be orthogonal to functoriality in this document.
}

\In{proposition}{
    Let $F : \cat C \to \cat D$ be an infrafunctor. Recall it is by definition a family of functions %
    \In{math}{
        (F_{A,B} : \cat C(A,B) \to \cat D(FA,FB))_{A,B}
    }%
    . Then $F_{A,B}$ is natural in $A$ \emph{or} $B$ if and only if $F$ is composition-compatible.
}
    

\subsection{Natural Connectors}

In string diagrams, a natural transformation is a connector of two tubes
\In{align*}{
    \strdgathered{
        \strdfork{
            \strdmid,\strdrlabel{}
            ,\strdmid,\strdrlabel{}
        }
        ,\strdtube{
            \strdtubestartline
            ,\strdtubemid,\strdat{Left}{\strdoverlabel{G}}
            ,\strdtwincell{\tau}
            ,\strdtubemid,\strdat{Left}{\strdoverlabel{F}}
            ,\strdtubeendline
        }
    }
}
because the naturality states a node can travel between tubes
\In{align*}{
    \strdgathered{
        \strdfork{
            \strdmid,\strdrlabel{B}
            ,\strdmid,\strdrlabel{}
            ,\strdcell{f}
            ,\strdmid,\strdrlabel{A}
        }
        ,\strdtube{
            \strdtubestartline
            ,\strdtubemid
            ,\strdtwincell{\tau}
            ,\strdtubemid
            ,\strdtubemid
            ,\strdtubeendline
        }
    } =
    \strdgathered{
        \strdfork{
            \strdmid,\strdrlabel{B}
            ,\strdcell{f}
            ,\strdmid,\strdrlabel{}
            ,\strdmid,\strdrlabel{A}
        }
        ,\strdtube{
            \strdtubestartline
            ,\strdtubemid
            ,\strdtubemid
            ,\strdtwincell{\tau}
            ,\strdtubemid
            ,\strdtubeendline
        }
    }
}
This inspires you to assign
\In{align*}{
    \strdgathered{
        \strdfork{
            \strdmid,\strdrlabel{B}
            ,\strdcell{f}
            ,\strdmid,\strdrlabel{A}
        }
        ,\strdtube{
            \strdtubestartline
            ,\strdtubemid%,\strdat{Left}{\strdoverlabel{G}}
            ,\strdtwincell{\tau}
            ,\strdtubemid%,\strdat{Left}{\strdoverlabel{F}}
            ,\strdtubeendline
        }
    }
}

\definitionof{Vertical Composition}{
    Given three functors
    \In{align*}{
        F,G,H : \cat{C} \to \cat{D}
    }
    and two natural transformations
    \In{align*}{
        \tau &: F \to G\\
        \sigma &: G \to H
    }
    the \emph{vertical composition} of $\tau$ and $\sigma$
    \In{align*}{
        \sigma \circ \tau : F \to H
    }
    is defined by
    \In{align*}{
        \strdgathered{
            \strdfork{
                \strdmid,\strdrlabel{}
                ,\strdmid,\strdrlabel{}
                ,\strdmid,\strdrlabel{}
            }
            ,\strdtube{
                \strdtubestartline
                ,\strdtubemid,\strdat{Left}{\strdoverlabel{H}}
                ,\strdtwincell{\sigma}
                ,\strdtubemid,\strdat{Left}{\strdoverlabel{G}}
                ,\strdtwincell{\tau}
                ,\strdtubemid,\strdat{Left}{\strdoverlabel{F}}
                ,\strdtubeendline
            }
        }
    }
}

\definitionof{Horizontal Composition}{
    Given four functors
    \In{align*}{
        F,G &: \cat{A} \to \cat{B}\\
        H,K &: \cat{B} \to \cat{C}\\
    }
    and two natural transformations
    \In{align*}{
        \tau &: F \to G\\
        \sigma &: H \to K
    }
    the \emph{horizontal composition} of $\tau$ and $\sigma$
    \In{align*}{
        \sigma \tau : H \circ F \to K \circ G 
    }
    is defined by
    \In{align*}{
        \strdgathered{
            \strdfork{
                \strdmid,\strdrlabel{}
                ,\strdmid,\strdrlabel{}
            }
            ,\strdtube{
                \strdtubestart
                ,\strdfork{
                    \strdtubemid,\strdat{Left}{\strdoverlabel{G}}
                    ,\strdtwincell{\tau}
                    ,\strdtubemid,\strdat{Left}{\strdoverlabel{F}}
                }
                ,\strdtubeextend{1}{1}
                ,\strdcutoff\strdtubebound
                ,\strdtubemid,\strdat{Left}{\strdoverlabel{K}}
                ,\strdtwincell{\sigma}
                ,\strdtubemid,\strdat{Left}{\strdoverlabel{H}}
                ,\strdcutoff\strdtubebound
            }
        }
    }
}
You can easily check the naturality. Travel by car ferry.


\definitionof{Identity Natural Transformation}{
    Given a functor
    \In{align*}{
        F &: \cat{C} \to \cat{D}
    }
    the \emph{identity natural transformation}
    \In{align*}{
        \operatorname{id} _F : F \to F
    }
    is defined by
    \In{align*}{
        \strdgathered{
            \strdfork{
                \strdmid,\strdrlabel{}
            }
            ,\strdtube{
                \strdtubestartline
                ,\strdtubemid
                ,\strdtubeat{Mid}{\strdtwincell{\operatorname{id}}}
                ,\strdtubeendline
            }
        } \mydefeq\,
        \strdgathered{
            \strdfork{
                \strdmid,\strdrlabel{}
            }
            ,\strdtube{
                \strdtubestartline
                ,\strdtubemid
                ,\strdtubeendline
            }
        }
    }
}

\definitionof{Whiskering}{
    A \emph{whikering} is a horizontal composition with identity natural transformations:
    \In{align*}{
        \strdgathered{
            \strdfork{
                \strdmid,\strdrlabel{}
                ,\strdmid,\strdrlabel{}
            }
            ,\strdtube{
                \strdtubestart
                ,\strdfork{
                    \strdtubemid
                    ,\strdtwincell{\tau}
                    ,\strdtubemid
                }
                ,\strdtubeextend{1}{1}
                ,\strdcutoff\strdtubebound
                ,\strdtubemid
                ,\strdtubemid
                ,\strdcutoff\strdtubebound
            }
        }, 
        \strdgathered{
            \strdfork{
                \strdmid,\strdrlabel{}
                ,\strdmid,\strdrlabel{}
            }
            ,\strdtube{
                \strdtubestart
                ,\strdfork{
                    \strdtubemid
                    ,\strdtubemid
                }
                ,\strdtubeextend{1}{1}
                ,\strdcutoff\strdtubebound
                ,\strdtubemid
                ,\strdtwincell{\sigma}
                ,\strdtubemid
                ,\strdcutoff\strdtubebound
            }
        }
    }
}

\definitionof{Natural Isomorphism}{
    A \emph{natural isomorphism} is a pair of natural transformations
    \In{align*}{
        \tau &: F \to G\\
        \sigma &: G \to F
    }
    satisfying the \emph{invertibilty}:
    \In{align*}{
        \strdgathered{
            \strdfork{
                \strdmid,\strdrlabel{}
                ,\strdmid,\strdrlabel{}
                ,\strdmid,\strdrlabel{}
            }
            ,\strdtube{
                \strdtubestartline
                ,\strdtubemid
                ,\strdtwincell{\sigma}
                ,\strdtubemid
                ,\strdtwincell{\tau}
                ,\strdtubemid
                ,\strdtubeendline
            }
        } & =\
        \strdgathered{
            \strdfork{
                \strdmid,\strdrlabel{}
            }
            ,\strdtube{
                \strdtubestartline
                ,\strdtubemid
                ,\strdtubeendline
            }
        } \\
        \strdgathered{
            \strdfork{
                \strdmid,\strdrlabel{}
                ,\strdmid,\strdrlabel{}
                ,\strdmid,\strdrlabel{}
            }
            ,\strdtube{
                \strdtubestartline
                ,\strdtubemid
                ,\strdtwincell{\tau}
                ,\strdtubemid
                ,\strdtwincell{\sigma}
                ,\strdtubemid
                ,\strdtubeendline
            }
        } & =\
        \strdgathered{
            \strdfork{
                \strdmid,\strdrlabel{}
            }
            ,\strdtube{
                \strdtubestartline
                ,\strdtubemid
                ,\strdtubeendline
            }
        }
    }
}
The same symbol is often used for the pair. 

\In{proposition}{
    For any natural transformation $\tau$,
    \In{align*}{
        (\forall A)(\tau_A : \text{invertible})
    }
    is enough to build the other natural $\sigma$.
}

\definitionof{Functor Category}{
    Given a small category $\cat{C}$ and a category $\cat{D}$, %
    the functor category
    $
        [\cat{C},\cat{D}]
    $
    is a category whose objects are functors from $\cat{C}$ to $\cat{D}$ and %
    whose morphisms are natural transformations:
    \In{align*}{
        \strdgathered{
            \strdfork{
                \strdmid,\strdrlabel{H}
                ,\strdcell{\sigma}
                ,\strdmid,\strdrlabel{G}
                ,\strdcell{\tau}
                ,\strdmid,\strdrlabel{F}
            }
        }
    }
    , where vertical compositions join the strings. 
}

\In{definition}{
    For the later use, define a lambda-tasted form for a set of natural transformations:
    \In{align*}{
        \operatorname{Nat}_A(FA,GA) \mydefeq \operatorname{Nat}(F,G) \mydefeq [\cat{C},\cat{D}](F,G)
    }
}

