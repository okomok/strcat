\section{Functors}

\subsection{Definition}
Given categories $\cat{C}$ and $\cat{D}$, a \textit{functor}
\In{align*}{
    F : \cat{C} \to \cat{D}
}
consists of:
\In{enumerate}{
\item a family of objects
\In{align*}{
    (FA \in \operatorname{Ob}\cat{D}) _ {A \in \operatorname{Ob}\cat{C}}
}
\item a family of morphisms
\In{align*}{
    (F(f) \in \cat{D}(FA,FB)) _ {f \in \cat{C}(A,B)}
}
}
satisfying the \textit{functoriality}:
\In{enumerate}{
\item \textit{Composition-compatibility}: for any $f \in \cat{C}(A,B)$ and $g \in \cat{C}(B,C)$,
    \In{align*}{
        F(g \circ f) = F(g) \circ F(f)
    }
\item \textit{Unit-compatibility}: for any $A \in \operatorname{Ob}\cat{C}$,
    \In{align*}{
        F(\operatorname{id}_A) = \operatorname{id}_{FA}
    }
}
An \textit{infrafunctor} is a structure without functoriality.


\subsection{Functorial Boxes}

In string diagrams, a functor is represented as an open box

\In{align*}{
    \strdgathered{
        \strdfork{
            \strdmid,\strdrlabel{B}
            ,\strdcell{f}
            ,\strdmid,\strdrlabel{A}
        }
        ,\strdtube{
            \strdtubestartline
            ,\strdydouble\strdtubemid
            ,\strdat{Left}{\strdoverlabel{F}} 
            ,\strdtubeendline
        }
    } \mydefeq
    \strdgathered{
        \strdmid,\strdrlabel{FB}
        ,\strdbox{1}{1}{
            \strdmid,\strdrlabel{B}
            ,\strdcell{f}
            ,\strdmid,\strdrlabel{A}
        }
        ,\strdat{Left}{\strdoverlabel{F}}
        ,\strdmid,\strdrlabel{FA}
    }
}
Placeholders make it simple:
\In{align*}{
    \strdgathered{
        \strdfork{
            \strdmid,\strdrlabel{}
            ,\strdcell{?}
            ,\strdmid,\strdrlabel{}
        }
        ,\strdtube{
            \strdtubestartline
            ,\strdydouble\strdtubemid
            ,\strdat{Left}{\strdoverlabel{F}} 
            ,\strdtubeendline
        }
    } 
}
One can check the functoriality ensures any string like
\In{align*}{
    \strdgathered{
        \strdfork{
            \strdmid,\strdrlabel{C}
            ,\strdcell{g}
            ,\strdmid,\strdrlabel{B}
            ,\strdcell{f}
            ,\strdmid,\strdrlabel{A}
        }
        ,\strdtube{
            \strdtubestartline
            ,\strdyscale{3}\strdtubemid
            ,\strdtubeendline
        }
    }
}
be unambiguous. "Join then box" is the same as "Box then join".


\subsection{Composite Functors}

For any two functors 
\In{align*}{
    F &: \cat{A} \to \cat{B}\\
    G &: \cat{B} \to \cat{C}
}
, the \textit{composite functor} of $F$ and $G$
\In{align*}{
    G \circ F : \cat{A} \to \cat{C}
}
is depicted as 
\In{align*}{
    \strdgathered{
        \strdfork{
            \strdmid
            ,\strdcell{\Identity{?}}
            ,\strdmid
        }
        ,\strdtubestart
        ,\strdfork{
            \strdtube{
                \strdtubemid
                ,\strdat{Left End}{\strdoverlabel{F}} 
                ,\strdtubemid
            }
        }
        ,\strdtubeextend{1}{1}
        ,\strdtube{
            \strdcutoff\strdtubebound
            ,\strdtubemid
            ,\strdat{Left End}{\strdoverlabel{G}} 
            ,\strdtubemid
            ,\strdtubeendline
        }
    }
}
One can easily check the functoriality.


\subsection{Identity Functors}

An \textit{idenity functor}
\In{align*}{
    \operatorname{Id}_{\cat{C}} : \cat{C} \to \cat{C}
}
is depicted as a transparent box
\In{align*}{
    \strdgathered{
        \strdfork{
            \strdmid,\strdrlabel{}
            ,\strdcell{?}
            ,\strdmid,\strdrlabel{}
        }
        ,\strdtube{
            \strdtubestartline
            ,\strdydouble\strdtubemid
            ,\strdat{Left}{\strdoverlabel{\operatorname{Id}}} 
            ,\strdtubeendline
        }
    } \mydefeq
    \strdgathered{
        \strdmid,\strdrlabel{}
        ,\strdcell{?}
        ,\strdmid,\strdrlabel{}
    }
}
The functoriality is trivial.


\subsection{Contravariant Functors}

A functor whose domain is an opposite category
\In{align*}{
    F : \cat{C}^{\operatorname{op}} \to \cat{D}
}
is called \textit{contravariant}, while a normal functor is called \textit{covariant}.
A contravariant functor is depicted as
\In{align*}{
    \strdgathered{
        \strdfork{
            \strdmid
            ,\strdcell{\strdupsidedown{?}}
            ,\strdmid
        }
        ,\strdtube{
            \strdtubestartline
            ,\strdtubemid
            %,\strdat{Left End}{\strdoverlabel{F}} 
            ,\strdtubemid
            ,\strdtubeendline
        }
    }
}


\subsection{Binary Functors}

A functor whose domain is a product category
\In{align*}{
    F : \cat{A} \times \cat{B} \to \cat{C}
}
is called a \textit{binary functor} or \textit{bifunctor}.
Spelling out the definition of functoriality, you get
TODO



\subsection{Hom Functors}
TODO
