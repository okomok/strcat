\section{Category of Sets}

\strdexternalname{category_of_sets}


\subsection{The Definition}

\definitionof{Category of Sets}{
    The \emph{category of sets} $\catSet$ is a category whose objects are sets and whose morphisms are functions:
    \In{align*}{
        \strdexternal\strdgathered{
            \strdmid,\strdrlabel{Z}
            ,\strdcell{g}
            ,\strdmid,\strdrlabel{Y}
            ,\strdcell{f}
            ,\strdmid,\strdrlabel{X}
        }
    }
    where strings are joined by the function composition. 
}

A category is essentially one-dimensional so far: the vertical composition only. %
Here we introduce the horizontal composition for \catSet.

\definitionof{Monoidal Category of Sets}{
    Parallel strings are defined by
    \In{align*}{
        \strdexternal\strdgathered{
            \strdfork{
                \strdxscale{0.7}\strdljump
                ,\strdmid,\strdrlabel{X}
            }
            ,\strdfork{
                \strdxscale{0.7}\strdrjump
                ,\strdmid,\strdrlabel{X\myprime}
            }
        } \mydefeq\,
        \strdexternal\strdgathered{
            \strdmid,\strdrlabel{X {\times} X\myprime}
        }
    }
    The \emph{horizontal composition} of functions is defined by
    \In{align*}{
        \strdexternal\strdgathered{
            \strdfork{
                \strdxscale{0.7}\strdljump
                ,\strdmid,\strdrlabel{Y}
                ,\strdcell{f}
                ,\strdmid,\strdrlabel{X}
            }
            ,\strdfork{
                \strdxscale{0.7}\strdrjump
                ,\strdmid,\strdrlabel{Y\myprime}
                ,\strdcell{f\myprime}
                ,\strdmid,\strdrlabel{X\myprime}
            }
        } \mydefeq\,
        \Lambda _{x,x\myprime} (f(x),f\myprime(x\myprime))
    }
    Strings for the singleton set $\lbrace \ast\rbrace$ is omitted so that an element of a set is represented as
    \In{align*}{
        \strdexternal\strdgathered{
            \strdmid,\strdrlabel{X}
            ,\strdcell{x}
        }
    }
}

One can check any string diagram built upon these definitions is unambiguous due to the trivial bijections
\In{align*}{
    X \times (X\myprime \times X\myprime\myprime) &\cong (X \times X\myprime) \times X\myprime\myprime\\
    X \times \lbrace\ast\rbrace &\cong X
}
Informally such two-dimensional categories are called \emph{monoidal}.


\subsection{Hom-Set Bands}

Given a category $\cat{C}$, a special string, a \emph{band}, is introduced for hom-sets:
\In{align*}{
    \strdexternal\strdgathered{
        \strdtubeof{}{
            \strdtubestart%line
            ,\strdtubemid,\strdbackground{strd roadcolor}
            ,\strdat{Left}{\strdrlabel{B}}
            ,\strdat{Right}{\strdllabel{A}}
            %,\strdtubeendline
        }
    } \mydefeq\,
    \strdexternal\strdgathered{
        \strdmid,\strdrlabel{\cat{C}(A{,}B)}
    }
}
A space-saving form is depicted as
\In{align*}{
    \strdexternal\strdgathered{
        \strdroad\strdmid,\strdroadlabel{B}{A}
    } 
}

\In{remark}{
    Note that the order of objects is flipped. %
    This is resulting from an unfortunate convention that one write ``$b = h(a)$" %
    but not ``$h : B \leftarrow A$". By the way, ``$B^A$" is fine.
}

The composition of morphisms can be depicted as
\In{align*}{
    \strdexternal\strdgathered{
        \strdtubestart
        ,\strdtubeextend{1}{1}
        ,\strdyhalf\strdtubemid,\strdbackground{strd roadcolor}
        ,\strdfork{
            \strdtubepartleft{1},\strdbackground{strd roadcolor}
        }
        ,\strdfork{
            \strdtubepartright{1},\strdbackground{strd roadcolor}
        }
    }\ \ \text{or}\ \
    \strdexternal\strdgathered{
        \strdroad\strdmid
        ,\strdfork{
            \strdroad\strdleft
        }
        ,\strdroad\strdright
    }
}
\mynewline
Identity morphisms can be depicted as
\In{align*}{
    \strdexternal\strdgathered{
        \strdtubestart
        ,\strdtubeclose,\strdbackground{strd roadcolor}
    }\ \ \text{or}\ \
    \strdexternal\strdgathered{
        \strdroad\strdmid
        ,\strddeadend
    }
}
As an exercise, write down the associativity and unitality using these diagrams. 

\definitionof{Naming}{
    We don't distinguish the following two forms
    \In{align*}{
        \strdexternal\strdgathered{
            \strdmid,\strdrlabel{B}
            ,\strdcell{h}
            ,\strdmid,\strdrlabel{A}
        } \,\sim\,
        \strdexternal\strdgathered{
            \strdroad\strdmid,\strdroadlabel{B}{A}
            ,\strdcell{h}
        }
    }
    We say $h$ in \cat C is \emph{named} in \catSet.
}

% Needs extranaturality
% \In{proposition}{
%     Naming {preserves naturalty}, meaning that given a family of morphisms 
%     $(f_X : FX \to GX)_X$ with infrafunctors $F$ and $G$, %
%     $f_X$ is natural in $X$ if and only if
%     $\operatorname{name}(f_X): \catSing \to \catSet(FX,GX)$ is natural in $X$. %
%     So does unnaming.
% }

\definitionof{Hom-Functor}{
    Hom-sets can be extended to the \emph{Hom-functor}
    \In{align*}{
        \Lambda_{A,B} \cat{C}(A,B) &: \cat{C}^{\operatorname{op}} \times \cat{C} \to \catSet \text{ or shortly}\\
        \cat{C}(\texttt{-},\texttt{+}) &: \cat{C}^{\operatorname{op}} \times \cat{C} \to \catSet \text{ or shortly}\\
        \operatorname{Hom}_{\cat{C}} &: \cat{C}^{\operatorname{op}} \times \cat{C} \to \catSet 
    }
    defined by
    \In{align*}{
        \strdexternal\strdgathered{
            \strdtubestart%line
            ,\strdtubemid,\strdbackground{strd roadcolor}
            ,\strdat{Right}{\strdllabel{A\myprime}}
            ,\strdat{Left}{\strdrlabel{B\myprime}}
            ,\strdbox{2}{2}{
                \strdtubeof{}{
                    \strdtubestart
                    ,\strdtubemid
                    ,\strdat{Right}{\strdrlabel{B\myprime}}
                    ,\strdat{Left}{\strdrlabel{A\myprime}}
                    ,\strdat{Right End}{\strdcell{b}}
                    ,\strdat{Left End}{\strdcell{\strdupsidedown{a}}}
                    ,\strdtubemid
                    ,\strdat{Right}{\strdrlabel{B}}
                    ,\strdat{Left}{\strdrlabel{A}}
                }
            },\strdat{Left}{\strdoverlabel{\strdlvert{\operatorname{Hom}}}}
            ,\strdtubestart
            ,\strdtubemid,\strdbackground{strd roadcolor}
            ,\strdat{Right}{\strdllabel{A}}
            ,\strdat{Left}{\strdrlabel{B}}
        } \mydefeq\,
        \strdexternal\strdgathered{
            \strdmid,\strdrlabel{B\myprime}
            ,\strdcell{b}
            ,\strdmid,\strdrlabel{B}
            ,\strdcell{\texttt{?}}
            ,\strdmid,\strdrlabel{A}
            ,\strdcell{a}
            ,\strdmid,\strdrlabel{A\myprime}
        }
    }
    where the world in the box is product category $\cat{C}^{\operatorname{op}} \times \cat{C}$.
}

This definition inspires us to depict hom-functors as
\In{align*}{
    \strdexternal\strdgathered{
        \strdtubeof{}{
            \strdtubestart%line
            ,\strdtubemid,\strdbackground{strd roadcolor}
            ,\strdat{Left End}{\strdcell{\texttt{2}}}
            ,\strdat{Right End}{\strdcell{\strdupsidedown{\texttt{1}}}}
            ,\strdtubemid,\strdbackground{strd roadcolor}
        }
    } \ \ \text{or}\ \
    \strdexternal\strdgathered{
        \strdyhalf{\strdroad\strdmid}
        ,\strdfork{
            \strdroad\strdleft
            ,\strdcell{\texttt{2}}
        }
        ,\strdfork{
            \strdroad\strdright
            ,\strdcell{\texttt{1}}
        }
        ,\strdroad\strdmid
        ,\strdyhalf{\strdroad\strdmid}
    }
}
that looks topologically equivalent.

\definitionof{Unary Hom-Functor}{
    Due to \cref{Partial Application},
    \In{align*}{
        \cat{C}(A,\texttt{+}) &: \cat{C} \to \catSet\\
        \cat{C}(\texttt{-},B) &: \cat{C}^{\operatorname{op}} \to \catSet
    }
    are respectively depicted as
    \In{align*}{
        \strdexternal\strdgathered{
            \strdtubeof{}{
                \strdtubestart%line
                ,\strdtubemid,\strdbackground{strd roadcolor}
                ,\strdat{Right End}{\strdllabel{A}}
                ,\strdat{Left End}{\strdcell{\texttt{?}}}
                ,\strdtubemid,\strdbackground{strd roadcolor}
            }
        },\ 
        \strdexternal\strdgathered{
            \strdtubeof{}{
                \strdtubestart%line
                ,\strdtubemid,\strdbackground{strd roadcolor}
                ,\strdat{Right End}{\strdcell{\strdupsidedown{\texttt{?}}}}
                ,\strdat{Left End}{\strdrlabel{B}}
                ,\strdtubemid,\strdbackground{strd roadcolor}
            }
        }
    }
}

\definitionof{Currying}{
    In particular case $\cat{C} = \catSet$, the \emph{curry bijection} %
    or shortly \emph{currying} is defined by
    \In{align*}{
        \mylhs{\catSet(A \times B,C)} &\cong \myrhs{\catSet(A,\catSet(B,C))} \\
        h &\mapsto (a \mapsto b \mapsto h(a,b)) \\
        ((a,b) \mapsto h(a)(b)) &\mymapsfrom h \\
        \strdexternal\strdgathered{
            \strdmid,\strdrlabel{C}
            ,\strdwidecell{1}{1}{h}
            ,\strdtubestart
            ,\strdtubemid
            ,\strdat{Left}{\strdrlabel{A}}
            ,\strdat{Right}{\strdrlabel{B}}
        } &\sim
        \strdexternal\strdgathered{
            \strdtubestart%line
            ,\strdtubemid,\strdbackground{strd roadcolor}
            ,\strdwidecell{1}{1}{h}
            ,\strdnone{\strdat{Left}{\strdllabel{C}}}
            ,\strdat{Left}{\strdrlabel{C}}
            ,\strdat{Right}{\strdllabel{B}}
            ,\strdmid,\strdrlabel{A}
        }
    }
}

We don't distinguish these two diagrams because %
the following two propositions ensure ``move the right-side leg up and down" works correct.

\In{proposition}{
    Currying is natural in all three variables.
}

\In{proposition}{
    Currying merges:
    \In{align*}{
        \strdexternal\strdgathered{
            \strdyhalf{\strdroad\strdmid}
            ,\strdfork{
                \strdxscale{1.7}{\strdroad\strdleft}
                ,\strdroadlaside
                ,\strdbox{1.5}{1.5}{
                    \strdmid
                    ,\strdcell{g}
                    ,\strdfork{
                        \strdright
                    }
                    ,\strdleft
                }
                ,\strdyhalf\strdmid
            }
            ,\strdfork{
                \strdxscale{1.7}{\strdroad\strdright}
                ,\strdroadlaside
                ,\strdbox{1.5}{1.5}{
                    \strdmid
                    ,\strdcell{f}
                    ,\strdfork{
                        \strdright
                    }
                    ,\strdleft
                }
                ,\strdyhalf\strdmid
            }
        } =
        \strdexternal\strdgathered{
            \strdyhalf{\strdroad\strdmid}
            ,\strdroadlaside
            ,\strdbox{2}{3}{
                \strdmid
                ,\strdcell{g}
                ,\strdfork{
                    \strdleft
                    ,\strdmid
                    ,\strdyhalf\strdmid
                }
                ,\strdright
                ,\strdcell{f}
                ,\strdfork{\strdright}
                ,\strdleft
            }
            ,\strdyhalf\strdmid
        }
    }
}

\In{remark}{
    It is equivalently naming to curry a function whose left-side leg is the singleton set.
}

\In{proposition}{
    Currying preserves naturality, meaning that given a family of functions %
    $(f_X : FX \times B \to GX)_X$ with infrafunctors $F$ and $G$,
    $f_X$ is natural in $X$ if and only if $\operatorname{curry}(f_X)$ is. So does uncurrying.
}

\In{remark}{
    In general, natural bijections have similar properties so that %
    you don't bother with proof of naturality (\cite{Kelly198204}). 
}

