\section{Category of Sets}

\subsection{Definition}

The \emph{category of sets} $\catSet$ is a category whose objects are sets and whose morphisms are functions:
\In{align*}{
    \strdgathered{
        \strdmid,\strdrlabel{Z}
        ,\strdcell{g}
        ,\strdmid,\strdrlabel{Y}
        ,\strdcell{f}
        ,\strdmid,\strdrlabel{X}
    }
}
, where nodes are joined by the function composition. 


\subsection{Monoidal Category of Sets}

A category is essentially one-dimensional so far: the vertical composition only. %
Here we introduce the horizontal composition for functions. Parallel strings are defined by
\In{align*}{
    \strdgathered{
        \strdfork{
            \strdxscale{0.7}\strdljump
            ,\strdmid,\strdrlabel{X}
        }
        ,\strdfork{
            \strdxscale{0.7}\strdrjump
            ,\strdmid,\strdrlabel{X\myprime}
        }
    } \mydefeq\,
    \strdgathered{
        \strdmid,\strdrlabel{X {\times} X\myprime}
    }
}
The horizontal composition is defined by
\In{align*}{
    \strdgathered{
        \strdfork{
            \strdxscale{0.7}\strdljump
            ,\strdmid,\strdrlabel{Y}
            ,\strdcell{f}
            ,\strdmid,\strdrlabel{X}
        }
        ,\strdfork{
            \strdxscale{0.7}\strdrjump
            ,\strdmid,\strdrlabel{Y\myprime}
            ,\strdcell{f\myprime}
            ,\strdmid,\strdrlabel{X\myprime}
        }
    } \mydefeq\,
    \Lambda _{x,x\myprime} (f(x),f\myprime(x\myprime))
}
Strings for the singleton set $\lbrace \ast\rbrace$ is omitted so that an element of a set is represented as
\In{align*}{
    \strdgathered{
        \strdmid,\strdrlabel{X}
        ,\strdcell{x}
    }
}
One can check any string diagram built upon these definitions is unambiguous thanks to the trivial bijections:
\In{align*}{
    X \times (X\myprime \times X\myprime\myprime) &\cong (X \times X\myprime) \times X\myprime\myprime\\
    X \times \lbrace\ast\rbrace &\cong X
}



\subsection{Hom-set Bands}

Given a category $\cat{C}$, a special string, a \emph{band}, is introduced for hom-sets:
\In{align*}{
    \strdgathered{
        \strdtubeof{}{
            \strdtubestart%line
            ,\strdtubemid,\strdbackground{strd roadcolor}
            ,\strdat{Left}{\strdrlabel{B}}
            ,\strdat{Right}{\strdllabel{A}}
            %,\strdtubeendline
        }
    } \mydefeq\,
    \strdgathered{
        \strdmid,\strdrlabel{\cat{C}(A{,}B)}
    }
}
A space-saving form is depicted as
\In{align*}{
    \strdgathered{
        \strdroad\strdmid,\strdroadlabel{B}{A}
    } 
}
Note that the order of objects is flipped. %
This is resulting from the unfortunate convention that we write $b = h(a)$ but not $h : B \leftarrow A$.
\mynewline
The composition of morphisms can be depicted as
\In{align*}{
    \strdgathered{
        \strdtubestart
        ,\strdtubeextend{1}{1}
        ,\strdyhalf\strdtubemid,\strdbackground{strd roadcolor}
        ,\strdfork{
            \strdtubepartleft{1},\strdbackground{strd roadcolor}
        }
        ,\strdfork{
            \strdtubepartright{1},\strdbackground{strd roadcolor}
        }
    }\ \ \text{or}\ \
    \strdgathered{
        \strdroad\strdmid
        ,\strdfork{
            \strdroad\strdleft
        }
        ,\strdroad\strdright
    }
}
\mynewline
Identity morphisms can be depicted as
\In{align*}{
    \strdgathered{
        \strdtubestart
        ,\strdtubeclose,\strdbackground{strd roadcolor}
    }\ \ \text{or}\ \
    \strdgathered{
        \strdroad\strdmid
        ,\strddeadend
    }
}
As an exercise, write down the associativity and unitality using these diagrams. 


\subsection{Currying}

In particular case $\cat{C} = \catSet$, there exists the \emph{curry bijection}
\In{align*}{
    \mylhs{\catSet(A \times B,C)} &\cong \myrhs{\catSet(A,\catSet(B,C))}\\
    \strdgathered{
        \strdmid,\strdrlabel{C}
        ,\strdwidecell{1}{1}{h}
        ,\strdtubestart
        ,\strdtubemid
        ,\strdat{Left}{\strdrlabel{A}}
        ,\strdat{Right}{\strdrlabel{B}}
    } &\sim
    \strdgathered{
        \strdtubestart%line
        ,\strdtubemid,\strdbackground{strd roadcolor}
        ,\strdwidecell{1}{1}{h}
        ,\strdnone{\strdat{Left}{\strdllabel{C}}}
        ,\strdat{Left}{\strdrlabel{C}}
        ,\strdat{Right}{\strdllabel{B}}
        ,\strdmid,\strdrlabel{A}
    }
}
We don't distinguish the two diagrams, for %
the naturality of the bijection ensures "Move the right-side leg up and down" works correct.
\mynewline
In case $A$ is the singleton set,
\In{align*}{
    \strdgathered{
        \strdmid,\strdrlabel{C}
        ,\strdcell{h}
        ,\strdmid,\strdrlabel{B}
    } \,\sim\,
    \strdgathered{
        \strdroad\strdmid,\strdroadlabel{C}{B}
        ,\strdcell{h}
    }
    % \strdgathered{
    %     \strdtubestart%line
    %     ,\strdtubemid,\strdbackground{strd roadcolor}
    %     ,\strdwidecell{1}{1}{h}
    %     ,\strdat{Left}{\strdrlabel{C}}
    %     ,\strdat{Right}{\strdllabel{B}}
    % }
}
is called a \emph{naming}, which turns a function to an element of function-sets.


\subsection{Hom Functors}

Hom-sets can be extended to a binary functor
\In{align*}{
    \Lambda_{A,B} \cat{C}(A,B) &: \cat{C}^{\operatorname{op}} \times \cat{C} \to \catSet \text{ or briefly}\\
    \cat{C}(\texttt{-},\texttt{+}) &: \cat{C}^{\operatorname{op}} \times \cat{C} \to \catSet \text{ or briefly}\\
    \operatorname{Hom}_{\cat{C}} &: \cat{C}^{\operatorname{op}} \times \cat{C} \to \catSet 
}
defined by
\In{align*}{
    \strdgathered{
        \strdtubestart%line
        ,\strdtubemid,\strdbackground{strd roadcolor}
        ,\strdat{Right}{\strdllabel{A\myprime}}
        ,\strdat{Left}{\strdrlabel{B\myprime}}
        ,\strdbox{2}{2}{
            \strdtubeof{}{
                \strdtubestart
                ,\strdtubemid
                ,\strdat{Right}{\strdrlabel{B\myprime}}
                ,\strdat{Left}{\strdrlabel{A\myprime}}
                ,\strdat{Right End}{\strdcell{b}}
                ,\strdat{Left End}{\strdcell{\strdupsidedown{a}}}
                ,\strdtubemid
                ,\strdat{Right}{\strdrlabel{B}}
                ,\strdat{Left}{\strdrlabel{A}}
            }
        },\strdat{Left}{\strdoverlabel{\strdlvert{\operatorname{Hom}}}}
        ,\strdtubestart
        ,\strdtubemid,\strdbackground{strd roadcolor}
        ,\strdat{Right}{\strdllabel{A}}
        ,\strdat{Left}{\strdrlabel{B}}
    } \mydefeq\,
    \strdgathered{
        \strdmid,\strdrlabel{B\myprime}
        ,\strdcell{b}
        ,\strdmid,\strdrlabel{B}
        ,\strdcell{\texttt{?}}
        ,\strdmid,\strdrlabel{A}
        ,\strdcell{a}
        ,\strdmid,\strdrlabel{A\myprime}
    }
}
Notice that the world in the box is the product category $\cat{C}^{\operatorname{op}} \times \cat{C}$.
\mynewline
This definition will inspire you to depict the hom functors as
\In{align*}{
    \strdgathered{
        \strdtubeof{}{
            \strdtubestart%line
            ,\strdtubemid,\strdbackground{strd roadcolor}
            ,\strdat{Left End}{\strdcell{\texttt{2}}}
            ,\strdat{Right End}{\strdcell{\strdupsidedown{\texttt{1}}}}
            ,\strdtubemid,\strdbackground{strd roadcolor}
        }
    } \ \ \text{or}\ \
    \strdgathered{
        \strdyhalf{\strdroad\strdmid}
        ,\strdfork{
            \strdroad\strdleft
            ,\strdcell{\texttt{2}}
        }
        ,\strdfork{
            \strdroad\strdright
            ,\strdcell{\texttt{1}}
        }
        ,\strdroad\strdmid
        ,\strdyhalf{\strdroad\strdmid}
    }
}
The partial applications
\In{align*}{
    \cat{C}(A,\texttt{+}) &: \cat{C} \to \catSet\\
    \cat{C}(\texttt{-},B) &: \cat{C}^{\operatorname{op}} \to \catSet
}
are respectively
\In{align*}{
    \strdgathered{
        \strdtubeof{}{
            \strdtubestart%line
            ,\strdtubemid,\strdbackground{strd roadcolor}
            ,\strdat{Right End}{\strdllabel{A}}
            ,\strdat{Left End}{\strdcell{\texttt{?}}}
            ,\strdtubemid,\strdbackground{strd roadcolor}
        }
    } \ , \
    \strdgathered{
        \strdtubeof{}{
            \strdtubestart%line
            ,\strdtubemid,\strdbackground{strd roadcolor}
            ,\strdat{Right End}{\strdcell{\strdupsidedown{\texttt{?}}}}
            ,\strdat{Left End}{\strdrlabel{B}}
            ,\strdtubemid,\strdbackground{strd roadcolor}
        }
    }
}




