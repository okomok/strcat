
\section{Adjunctions}

\definitionof{Adjunction}{
    Given two categories $\cat C$ and $\cat D$, an \emph{adjunction}
    \In{align*}{
        F \dashv G
    }
    consists of
    \In{enumerate}{
        \item \emph{left adjoint}: a functor $F : \cat C \to \cat D$ 
        \item \emph{right adjoint}: a functor $G : \cat D \to \cat C$ 
        \item \emph{adjunct}: a natural bijection 
        \In{align*}{
            (\cat D(FC, D) \cong \cat C(C,GD))_{C,D}
        }
    }
}

A nice consequence is that this bijectivity needs no boxes, expressed by natural transformations only.
\In{align*}{
    \strdgathered{
        \strdmid
        ,\strdwidecell{1}{1}{f}
        ,\strdfork{
            \strdmid
        }
        ,\strdtubestart
        ,\strdtube\strdtubemid,\strdat{Left}{\strdoverlabel{F}}
        ,\strdtubeendline
    } =
    \strdgathered{
        \strdmid
        ,\strdfork{
            \strdmid
            ,\strdmid
        }
        ,\strdtube{
            \strdtubestart
            ,\strdtwincell{\epsilon}
            ,\strdfork{
                \strdtubemid,\strdat{Left}{\strdoverlabel{G}}
                ,\strdwidecell{1}{1}{g}
            }
            ,\strdtubecap{1}{1}
            ,\strdtubemid,\strdat{Left}{\strdoverlabel{F}}
            ,\strdtubeendline
        }
    }
    \iff
    \Compose\strdgathered\strdinv{
        \strdmid
        ,\strdfork{
            \strdmid
            ,\strdmid
        }
        ,\strdtube{
            \strdtubestart
            ,\strdtwincell{\eta}
            ,\strdfork{
                \strdtubemid,\strdat{Left}{\strdoverlabel{F}}
                ,\strdwidecell{1}{1}{f}
            }
            ,\strdtubecap{1}{1}
            ,\strdtubemid,\strdat{Left}{\strdoverlabel{G}}
            ,\strdtubeendline
        }
    }
    =
    \Compose\strdgathered\strdinv{
        \strdmid
        ,\strdwidecell{1}{1}{g}
        ,\strdfork{
            \strdmid
        }
        ,\strdtubestart
        ,\strdtube\strdtubemid,\strdat{Left}{\strdoverlabel{G}}
        ,\strdtubeendline
    }
}
where
\In{align*}{
    \Compose\strdgathered\strdinv{
        \strdmid
        ,\strdfork{
            \strdmid
        }
        ,\strdtube{
            \strdtubestart
            ,\strdtwincell{\eta}
            ,\strdfork{
                \strdtubemid,\strdat{Left}{\strdoverlabel{F}}
            }
            ,\strdtubecap{1}{1},\strdat{Left}{\strdoverlabel{G}}
            ,\strdtubeendline
        }
    } &\mydefeq
    \Compose\strdgathered\strdinv{
        \strdmid
        ,\strdfork{
            \strdmid
            ,\strdmid
        }
        ,\strdbox{2}{2}{
            \strdtubestart
            ,\strdtube\strdtubemid
        },\strdat{Left}{\strdoverlabel{\cong}}
        ,\strdtube{
            \strdtubestart
            ,\strdfork{
                \strdtubemid,\strdat{Left}{\strdoverlabel{F}}
            }
            ,\strdtubeextend{1}{1}
            ,\strdtubemid,\strdat{Left}{\strdoverlabel{G}}
            ,\strdtubeendline
        }
    } \\
    \strdgathered{
        \strdmid
        ,\strdfork{
            \strdmid
        }
        ,\strdtube{
            \strdtubestart
            ,\strdtwincell{\epsilon}
            ,\strdfork{
                \strdtubemid,\strdat{Left}{\strdoverlabel{G}}
            }
            ,\strdtubecap{1}{1},\strdat{Left}{\strdoverlabel{F}}
            ,\strdtubeendline
        }
    } &\mydefeq
    \strdgathered{
        \strdmid
        ,\strdfork{
            \strdmid
            ,\strdmid
        }
        ,\strdbox{2}{2}{
            \strdtubestart
            ,\strdtube\strdtubemid
        },\strdat{Left}{\strdoverlabel{\cong}}
        ,\strdtube{
            \strdtubestart
            ,\strdfork{
                \strdtubemid,\strdat{Left}{\strdoverlabel{G}}
            }
            ,\strdtubeextend{1}{1}
            ,\strdtubemid,\strdat{Left}{\strdoverlabel{F}}
            ,\strdtubeendline
        }
    }
}
called the \emph{unit} and \emph{counit} respectively.

\In{proposition}{
    Given a functor $G: \cat D \to \cat C$, a family of natural bijections
    \In{align*}{
        ( (\cat C(C, GD) \cong \cat D(F_c, D) )_D )_C
    }
    is enough to construct the adjunction $F \dashv G$.
}

\In{strdproof}{
    Immediate by \cref{Parameterized Representations} with $H(C,D) \mydefeq \cat C(C,GD)$. 
}

\In{proposition}{[RAPL]
    \label{RAPL}
    Right adjoints preserve limits, meaning that given an adjunction %
    $F \dashv (G : \cat D \to \cat C)$ and a functor $T : \cat B \to \cat D$,
    \In{align*}{
        & (\text{lim}_X : \text{lim}T \to TX)_X : \text{limiting cone} \\
        \implies & (G(\text{lim}_X) : G\text{lim}T \to GTX)_X : \text{limiting cone}
    }
}

\InList{strdproof,align*}{
    &
    \strdgathered{
       \strdfork{
           \strdmid
           ,\strdmid
           ,\strdmid
       }
       ,\strdtube{
           \strdtubestart
           ,\strdfork{
               \strdtubemid
               ,\strdwidecell{1}{1}{\lim}
           }
           ,\strdfork{
               \strdtubeextend{1}{1},\strdcutoff\strdtubebound
               ,\strdtubemid
               ,\strdtubemid
               ,\strdwidecell{2}{2}{g}
           }
       }
    }
    =
    \strdgathered{
        \strdfork{
            \strdmid
        }
        ,\strdtubestart
        ,\strdfork{
            \strdtube\strdtubemid
        }
        ,\strdfork{
            \strdtubeextend{1}{1},\strdcutoff\strdtubebound
            ,\strdtube\strdtubemid
            ,\strdwidecell{2}{2}{v}
            ,\strdmid
        }
    }
    \\
    \iff &
    \strdgathered{
        \strdfork{
            \strdmid,\strdmid,\strdmid
        }
        ,\strdtubestartline
        ,\strdtube\strdtubemid
        ,\strdwidecell{1}{1}{\lim}
        ,\strdtubeextend{1}{1}
        ,\strdtwincell{\epsilon}
        ,\strdfork{
            \strdtube\strdtubemid
            ,\strdwidecell{2}{2}{g}
        }
        ,\strdtube{
            \strdtubecap{1}{1}
            ,\strdtubemid
            ,\strdtubeendline
        }
    }
    =
    \strdgathered{
        \strdfork{
            \strdmid,\strdmid,\strdmid
        }
        ,\strdtubestartline
        ,\strdtube\strdtubemid
        ,\strdfork{
            \strdtube\strdtubemid
        }
        ,\strdtubeextend{1}{1}
        ,\strdtwincell{\epsilon}
        ,\strdfork{
            \strdtube\strdtubemid
            ,\strdwidecell{2}{2}{v}
        }
        ,\strdtube{
            \strdtubecap{1}{1}
            ,\strdtubemid
            ,\strdtubeendline
        }
    }
    \\
    \iff &
    \strdgathered{
        \strdfork{
            \strdmid,\strdmid,\strdmid
        }
        ,\strdtubestart
        ,\strdtubejump
        ,\strdtubeextend{0}{0}
        ,\strdtwincell{\epsilon}
        ,\strdfork{
            \strdtube\strdtubemid
            ,\strdwidecell{1}{1}{g}
        }
        ,\strdtube{
            \strdtubecap{1}{1}
            ,\strdtubemid
            ,\strdtubeendline
        }
    }
    =
    \strdgathered{
        \strdmid
        ,\strdfork{
            \strdtubeof{strd frame}{
                \strdtubestart,\strdtubebound
                ,\strdtubesqcap{3}{3},\strdat{Left Corner}{\strdoverlabel{\lim}}
                ,\strdtubesqcup{3}{3}
            }
        }
        ,\strdfork{
            \strdmid,\strdmid,\strdmid
        }
        ,\strdtubestart
        ,\strdtube\strdtubemid
        ,\strdfork{
            \strdtube\strdtubemid
        }
        ,\strdtubeextend{1}{1}
        ,\strdtwincell{\epsilon}
        ,\strdfork{
            \strdtube\strdtubemid
            ,\strdwidecell{2}{2}{v}
        }
        ,\strdtube{
            \strdtubecap{1}{1}
            ,\strdtubemid
            ,\strdtubeendline
        }
    }
    \\
    \iff &
    \strdgathered{
        \strdfork{
            \strdmid,\strdmid
        }
        ,\strdtubestartline
        ,\strdtube\strdtubemid
        ,\strdwidecell{1}{1}{g}
    }
    =
    \strdgathered{
        \strdmid
        ,\strdfork{
            \strdtubeof{strd frame}{
                \strdtubestart,\strdtubebound
                ,\strdtubesqcap{3}{3},\strdat{Left Corner}{\strdoverlabel{\lim}}
                ,\strdtubesqcup{3}{3}
            }
        }
        ,\strdfork{
            \strdmid,\strdmid,\strdmid,\strdmid
        }
        ,\strdtubestart
        ,\strdtube\strdtubemid
        ,\strdfork{
            \strdtube\strdtubemid
        }
        ,\strdtubeextend{1}{1}
        ,\strdtwincell{\epsilon}
        ,\strdfork{
            \strdtube\strdtubemid
            ,\strdwidecell{2}{2}{v}
        }
        ,\strdtube{
            \strdtubecap{1}{1}
            ,\strdtubemid
            ,\strdtwincell{\eta}
            ,\strdinv{
                \strdtubecap{2}{2}
                ,\strdtubemid
                ,\strdtubemid
                ,\strdtubemid
                ,\strdtubeendline
            }
        }
    }
}

