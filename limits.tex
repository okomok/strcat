
\section{Limits}

\definitionof{Cone}{
    Given a functor $F : \cat A \to \cat B$, a cone of $F$ consists of
    \begin{enumerate}
        \item an object $B$ in \cat B
        \item a natural transformation $(v _ X : B \to FX) _ X$
    \end{enumerate}
}

\definitionof{Conicality}{
    We may explicitly call naturality of cones \emph{conicality}, which can be expressed as
    \In{align*}{
        \strdgathered{
            \strdfork{
                \strdmid
                ,\strdcell{f}
                ,\strdmid
            }
            ,\strdtubestartline
            ,\strdtube{
                \strdtubemid
                ,\strdtubemid
            }
            ,\strdwidecellof{strd widecellbase}{1}{1}{v}
            ,\strdmid,\strdrlabel{B}
        } =
        \strdgathered{
            \strdfork{
                \strdmid
            }
            ,\strdtubestartline
            ,\strdtube{
                \strdtubemid
            }
            ,\strdwidecellof{strd widecellbase}{1}{1}{v}
            ,\strdmid,\strdrlabel{B}
        }
    }
    like a magical box any morphism can appear from.
}

\In{remark}{
    Vertical and horizontal composition preserve conicality, a special case of naturality.
}

\definitionof{Limit}{
    Given a functor $F : \cat A \to \cat B$, a limit of $F$ is a pair of
    \In{enumerate}{
        \item an object in $\cat{B}$ denoted as $\text{lim} F$
        \item a natural bijection $(\cat{B}(B, \text{lim}F) \cong \catNat _ X(B,FX))_B$
    }
}

\definitionof{Limiting Cone}{
    The limit bijectivity, thanks to its naturality, can be expressed as
    \In{align*}{
        \strdgathered{
            \strdmid,\strdrlabel{\text{lim}F}
            ,\strdcell{h}
            ,\strdmid,\strdrlabel{B}
        } =
        \strdgathered{
            \strdmid,\strdrlabel{\text{lim}F}
            ,\strdbox{1.7}{1.7}{
                \strdfork{
                    \strdmid
                }
                ,\strdtubestart
                ,\strdtube\strdtubemid
                ,\strdwidecell{1}{1}{v}
            },\strdat{Left StartCorner}{\strdoverlabel{\text{lim}}}
           ,\strdmid,\strdrlabel{B}
        } \iff
        \strdgathered{
            \strdfork{
                \strdmid
            }
            ,\strdtubestartline
            ,\strdtube\strdtubemid
            ,\strdwidecell{1}{1}{\text{lim}}
            ,\strdmid,\strdrlabel{\text{lim} F}
            ,\strdcell{h}
            ,\strdmid,\strdrlabel{B}
        }
        =
        \strdgathered{
            \strdfork{
                \strdmid
            }
            ,\strdtubestartline
            ,\strdtube\strdtubemid
            ,\strdwidecell{1}{1}{v}
            ,\strdmid,\strdrlabel{B}
        }
    }
    where %
    \strdin{\strdpictureof{strd picture,baseline=(X.base)}}{\strdwidecellof{strd widecellbase,name=X}{1}{1}{\text{lim}}} %
    is a cone called a \emph{limiting cone} of F.
}

The following proposition justifies the notation $\text{lim}F$, \emph{the} limit of $F$.

\In{proposition}{
Limits are unique up to isomorphism.
}

\In{strdproof}{
    Immediate by \cref{Uniqueness of Representations}, because a limit is nothing but a contravariant representation
    \In{align*}{
        \text{rep} _ B \catNat_X(B,FX)
    }
}

\In{proposition}{
    A limiting cone is \emph{monic} meaning that
    \In{align*}{
        \strdgathered{
            \strdfork{
                \strdmid
            }
            ,\strdtubestartline
            ,\strdtube\strdtubemid
            ,\strdwidecell{1}{1}{\text{lim}}
            ,\strdmid,\strdrlabel{\text{lim} F}
            ,\strdcell{h}
            ,\strdmid,\strdrlabel{B}
        } =
        \strdgathered{
            \strdfork{
                \strdmid
            }
            ,\strdtubestartline
            ,\strdtube\strdtubemid
            ,\strdwidecell{1}{1}{\text{lim}}
            ,\strdmid,\strdrlabel{\text{lim} F}
            ,\strdcell{g}
            ,\strdmid,\strdrlabel{B}
        } \implies
        \strdgathered{
            \strdmid,\strdrlabel{\text{lim}F}
            ,\strdcell{h}
            ,\strdmid,\strdrlabel{B}
        } =
        \strdgathered{
            \strdmid,\strdrlabel{\text{lim}F}
            ,\strdcell{g}
            ,\strdmid,\strdrlabel{B}
        }
    }
}

\In{strdproof}{
    Immediate by the limit bijectivity.
}

\definitionof{Product}{
    In particular case the domain of a functor $F : \mathcal{A} \to \mathcal{B}$ is discrete, %
    the limit of $F$ is called the \emph{product }of $F$, denoted as $\textstyle\prod F$. %
}

\definitionof{Projection}{
    Spelling out the product bijectivity,
    \In{align*}{
        \strdgathered{
            \strdmid,\strdrlabel{\prod F}
            ,\strdcell{h}
            ,\strdmid,\strdrlabel{B}
        } =
        \strdgathered{
            \strdmid,\strdrlabel{\textstyle\prod F}
            ,\strdbox{1.7}{1.7}{
                \strdfork{
                    \strdmid
                }
                ,\strdtubestart
                ,\strdtube\strdtubemid
                ,\strdwidecell{1}{1}{v}
            },\strdat{Left StartCorner}{\strdoverlabel{\textstyle\prod}}
           ,\strdmid,\strdrlabel{B}
        } \iff
        \strdgathered{
            \strdfork{
                \strdmid
            }
            ,\strdtubestartline
            ,\strdtube\strdtubemid
            ,\strdwidecell{1}{1}{\pi}
            ,\strdmid,\strdrlabel{\textstyle\prod F}
            ,\strdcell{h}
            ,\strdmid,\strdrlabel{B}
        }
        =
        \strdgathered{
            \strdfork{
                \strdmid
            }
            ,\strdtubestartline
            ,\strdtube\strdtubemid
            ,\strdwidecell{1}{1}{v}
            ,\strdmid,\strdrlabel{B}
        }
    }
    where %
    \strdin{\strdpictureof{strd picture,baseline=(X.base)}}{\strdwidecellof{strd widecellbase,name=X}{1}{1}{\pi}} %
    is called the \emph{projection} of $F$. 
}

\In{remark}{
    Conicality has no concern here, because any family of the form
    \In{align*}{
        (v _ X : B \to FX) _ {X \in \catOb A}
    }
    is always natural in case $\cat A$ is discrete.
}

\In{example}{[Products in Sets]
    In case $F$ is a functor $X \to \cat Set$ with a set $X$(as a discrete category), %
    the product of $F$ is a set of dependent functions.
    \In{align*}{
        \textstyle\prod _{x} F(x) \cong \lbrace f \mid (f(x) \in F(x))_{x} \rbrace
    }
}

\definitionof{Dual}{
    Given a statement containing string diagrams, by flipping the diagrams upside down, %
    a corresponding statement is obtained. It is called the \emph{dual} of the original one.
}

\definitionof{Coproduct}{
    A \emph{coproduct} is a structure obtained from the product bijectivity flipped.
    \In{align*}{
        \Compose\strdgathered\strdinv{
            \strdmid,\strdrlabel{\coprod F}
            ,\strdcell{h}
            ,\strdmid,\strdrlabel{B}
        } =
        \Compose\strdgathered\strdinv{
            \strdmid,\strdrlabel{\textstyle\coprod F}
            ,\strdbox{1.7}{1.7}{
                \strdfork{
                    \strdmid
                }
                ,\strdtubestart
                ,\strdtube\strdtubemid
                ,\strdwidecell{1}{1}{v}
            },\strdat{Left StartCorner}{\strdoverlabel{\textstyle\coprod}}
           ,\strdmid,\strdrlabel{B}
        } \iff
        \Compose\strdgathered\strdinv{
            \strdfork{
                \strdmid
            }
            ,\strdtubestartline
            ,\strdtube\strdtubemid
            ,\strdwidecell{1}{1}{\operatorname{in}}
            ,\strdmid,\strdrlabel{\textstyle\coprod F}
            ,\strdcell{h}
            ,\strdmid,\strdrlabel{B}
        }
        =
        \Compose\strdgathered\strdinv{
            \strdfork{
                \strdmid
            }
            ,\strdtubestartline
            ,\strdtube\strdtubemid
            ,\strdwidecell{1}{1}{v}
            ,\strdmid,\strdrlabel{B}
        }
    }
}

\begin{remark}
Informally the dual makes a codomain opposite, while the variant does for a domain.
\end{remark}

\definitionof{Preservation of Limits}{
    Given a functor $F : \cat A \to \cat B$ and a limiting cone of $F$
    \In{align*}{
        (\text{lim}_X : \text{lim}F \to FX) _ X
    }
    a functor $G : \cat B \to \cat C$ \emph{preserves limits} of $F$ provided that
    \In{align*}{
        (G(\text{lim}_X) : G\text{lim}F \to GFX)_X
    } is a limiting cone of $G \circ F$.
}

In diagrams, $G$ is such that there exists some box $\texttt{!}$ satisfying
\In{align*}{
    \strdgathered{
        \strdfork{
            \strdmid
            ,\strdmid
        }
        ,\strdtubestartline
        ,\strdtube\strdtubemid,\strdat{Left}{\strdoverlabel{G}}
        ,\strdwidecell{1}{1}{g}
    }
    =
    \strdgathered{
        \strdfork{
            \strdmid
            ,\strdmid
        }
        ,\strdtubestart
        ,\strdtubeextend{1}{1},\strdcutoff\strdtubebound
        ,\strdtube\strdtubemid,\strdat{Left}{\strdoverlabel{G}}
        ,\strdbox{2.4}{2.4}{
            \strdtube{
                \strdfork{
                    \strdtubemid
                    ,\strdwidecell{2}{2}{v}
                }
                ,\strdtubeunextend{1}{1}
                ,\strdtubemid,\strdat{Left}{\strdoverlabel{F}}
            }
        },\strdat{Left StartCorner}{\strdoverlabel{\texttt{!}}}
        ,\strdmid
    } 
    \iff
    \strdgathered{
        \strdfork{
            \strdmid
            ,\strdmid
            ,\strdmid
        }
        ,\strdtubestart
        ,\strdtube{
            \strdfork{
                \strdtubemid,\strdat{Left}{\strdoverlabel{F}}
                ,\strdwidecell{1}{1}{\lim}
            }
            ,\strdtubeextend{1}{1},\strdcutoff\strdtubebound
            ,\strdtubemid,\strdat{Left End}{\strdoverlabel{G}}
            ,\strdtubemid
            ,\strdwidecell{2}{2}{g}
        }
    }
    =
    \strdgathered{
        \strdfork{\strdmid,\strdmid}
        ,\strdtubestart
        ,\strdtube{
            \strdfork{
                \strdtubemid,\strdat{Left}{\strdoverlabel{F}}
            }
            ,\strdtubeextend{1}{1},\strdcutoff\strdtubebound
            ,\strdtubemid,\strdat{Left}{\strdoverlabel{G}}
            ,\strdwidecell{2}{2}{v}
        }
    }
}

\In{proposition}{[HFPL]
    Hom-functors preserve limits, meaning that given a functor $F : \cat A \to \cat B$ %
    and an object $B$ in \cat B, the covariant hom-functor %
    $\cat B(B,\catph{+}) : \cat B \to \catSet$ preserves limits of $F$.
}

\InList{strdproof,align*}{
    &
    \strdgathered{
        \strdtubestart
        ,\strdtubemid,\strdat{Left End}{\strdcell{\scriptstyle\lim}},\strdbackground{strd roadcolor}
        ,\strdtubemid,\strdbackground{strd roadcolor},\strdat{Right Start}{\strdllabel{B}}
        ,\strdwidecell{1}{1}{g}
        ,\strdmid
    } =
    \strdgathered{
        \strdtubestart
        ,\strdtubemid,\strdbackground{strd roadcolor},\strdat{Right}{\strdllabel{B}}
        ,\strdwidecell{1}{1}{v}
        ,\strdmid
    } \\
    \iff &
    \strdgathered{
        \strdfork{
            \strdmid
        }
        ,\strdtubestartline
        ,\strdtube\strdtubemid
        ,\strdwidecell{1}{1}{\lim}
        ,\strdmid
        ,\strdcell{g(x)}
        ,\strdmid,\strdrlabel{B}
    } \underset{\forall x}{=}
    \strdgathered{
        \strdfork{
            \strdmid
        }
        ,\strdtubestartline
        ,\strdtube\strdtubemid
        ,\strdwidecell{1}{1}{v(x)}
        ,\strdmid,\strdrlabel{B}
    } \\
    \iff &
    \strdgathered{
        \strdmid
        ,\strdcell{g(x)}
        ,\strdtubestart
        ,\strdmid,\strdrlabel{B}
    } \,\underset{\forall x}{=}
    \strdgathered{
        \strdmid
        ,\strdbox{2}{2}{
            \strdfork{
                \strdmid
            }
            ,\strdtubestart
            ,\strdtube\strdtubemid
            ,\strdwidecell{1}{1}{v(x)}
        },\strdat{Left StartCorner}{\strdoverlabel{\lim}}
        ,\strdmid,\strdrlabel{B}
    }\\
    \iff &
    \strdgathered{
        \strdtubestart
        ,\strdtubemid,\strdbackground{strd roadcolor},\strdat{Right}{\strdllabel{B}}
        ,\strdwidecell{1}{1}{g}
        ,\strdmid
    } =
    \strdgathered{
        \strdbox{3}{3}{
            \strdmid
            ,\strdat{Start}{
                \strdtubestart
                ,\strdinv{
                    \strdxscale{1.5}\strdtuberjump
                    ,\strdtubemid,\strdbackground{strd roadcolor}
                    ,\strdat{Right}{\strdllabel{B}}
                }
            }
            ,\strdbox{2}{2}{
                \strdfork{
                    \strdmid
                }
                ,\strdtubestart
                ,\strdtube\strdtubemid
                ,\strdwidecell{1}{1}{v(x)}
            },\strdat{Left StartCorner}{\strdoverlabel{\lim}}
            ,\strdmid,\strdrlabel{B}
        },\strdat{Left}{\strdoverlabel{\Lambda_x}}
        ,\strdmid
    }
}

