
\section{Limits}

\subsection{Cones}

Given a functor $F : \cat A \to \cat B$, a natural transformation of the form
\In{align*}{
    (v _ X : B \to FX) _ X
}
is called a \emph{cone} of $F$. The naturality(\emph{conicality}) can be expressed as
\In{align*}{
    \strdgathered{
        \strdfork{
            \strdmid
            ,\strdcell{f}
            ,\strdmid
        }
        ,\strdtubestartline
        ,\strdtube{
            \strdtubemid
            ,\strdtubemid
        }
        ,\strdwidecellof{strd widecellbase}{1}{1}{v}
        ,\strdmid,\strdrlabel{B}
    } =
    \strdgathered{
        \strdfork{
            \strdmid
        }
        ,\strdtubestartline
        ,\strdtube{
            \strdtubemid
        }
        ,\strdwidecellof{strd widecellbase}{1}{1}{v}
        ,\strdmid,\strdrlabel{B}
    }
}
like a magical box any morphism can appear from.


\subsection{The Definition}

Given a functor $F : \cat A \to \cat B$, a limit of $F$ is a pair of
\In{enumerate}{
    \item an object in $\cat{B}$ denoted as $\text{lim} F$
    \item a natural bijection $(\cat{B}(B, \text{lim}F) \cong \catNat _ X(B,FX))_B$
}
The naturality allows the bijectivity to be expressed as
\In{align*}{
    \strdgathered{
        \strdmid,\strdrlabel{\text{lim}F}
        ,\strdcell{h}
        ,\strdmid,\strdrlabel{B}
    } =
    \strdgathered{
        \strdmid,\strdrlabel{\text{lim}F}
        ,\strdbox{1.7}{1.7}{
            \strdfork{
                \strdmid
            }
            ,\strdtubestart
            ,\strdtube\strdtubemid
            ,\strdwidecell{1}{1}{v}
        },\strdat{Left StartCorner}{\strdoverlabel{\text{lim}}}
       ,\strdmid,\strdrlabel{B}
    } \iff
    \strdgathered{
        \strdfork{
            \strdmid
        }
        ,\strdtubestartline
        ,\strdtube\strdtubemid
        ,\strdwidecell{1}{1}{\text{lim}}
        ,\strdmid,\strdrlabel{\text{lim} F}
        ,\strdcell{h}
        ,\strdmid,\strdrlabel{B}
    }
    =
    \strdgathered{
        \strdfork{
            \strdmid
        }
        ,\strdtubestartline
        ,\strdtube\strdtubemid
        ,\strdwidecell{1}{1}{v}
        ,\strdmid,\strdrlabel{B}
    }
}
, where \!\raisebox{0.25ex}{$\strdgathered{\strdwidecell{1}{1}{\text{lim}}}$} is called the \emph{limiting cone} of $F$.
\mynewline
A limit is nothing but a contravariant representation
\In{align*}{
    \text{rep} _ B \catNat_X(B,FX)
}
so that limits are unique up to isomorphism.
\begin{proposition}\label{l.c.m}
A limiting cone is \emph{monic} meaning that
\In{align*}{
    \strdgathered{
        \strdfork{
            \strdmid
        }
        ,\strdtubestartline
        ,\strdtube\strdtubemid
        ,\strdwidecell{1}{1}{\text{lim}}
        ,\strdmid,\strdrlabel{\text{lim} F}
        ,\strdcell{h}
        ,\strdmid,\strdrlabel{B}
    } =
    \strdgathered{
        \strdfork{
            \strdmid
        }
        ,\strdtubestartline
        ,\strdtube\strdtubemid
        ,\strdwidecell{1}{1}{\text{lim}}
        ,\strdmid,\strdrlabel{\text{lim} F}
        ,\strdcell{g}
        ,\strdmid,\strdrlabel{B}
    } \implies
    \strdgathered{
        \strdmid,\strdrlabel{\text{lim}F}
        ,\strdcell{h}
        ,\strdmid,\strdrlabel{B}
    } =
    \strdgathered{
        \strdmid,\strdrlabel{\text{lim}F}
        ,\strdcell{g}
        ,\strdmid,\strdrlabel{B}
    }
}
\end{proposition}


\subsection{Products}

In particular case the domain of a functor $F : \mathcal{A} \to \mathcal{B}$ is discrete, %
the limit of $F$ is called the \emph{product }of $F$ denoted as $\textstyle\prod F$. Spelling out the bijectivity,
\In{align*}{
    \strdgathered{
        \strdmid,\strdrlabel{\prod F}
        ,\strdcell{h}
        ,\strdmid,\strdrlabel{B}
    } =
    \strdgathered{
        \strdmid,\strdrlabel{\textstyle\prod F}
        ,\strdbox{1.7}{1.7}{
            \strdfork{
                \strdmid
            }
            ,\strdtubestart
            ,\strdtube\strdtubemid
            ,\strdwidecell{1}{1}{v}
        },\strdat{Left StartCorner}{\strdoverlabel{\textstyle\prod}}
       ,\strdmid,\strdrlabel{B}
    } \iff
    \strdgathered{
        \strdfork{
            \strdmid
        }
        ,\strdtubestartline
        ,\strdtube\strdtubemid
        ,\strdwidecell{1}{1}{\pi}
        ,\strdmid,\strdrlabel{\textstyle\prod F}
        ,\strdcell{h}
        ,\strdmid,\strdrlabel{B}
    }
    =
    \strdgathered{
        \strdfork{
            \strdmid
        }
        ,\strdtubestartline
        ,\strdtube\strdtubemid
        ,\strdwidecell{1}{1}{v}
        ,\strdmid,\strdrlabel{B}
    }
}
, where \!\raisebox{0.1ex}{$\strdgathered{\strdwidecell{1}{1}{\pi}}$} is called the \emph{projection} of $F$. %
Note that the conicality has no concern, because any family of the form
\In{align*}{
    (v _ X : B \to FX) _ {X \in \catOb A}
}
is always natural in case $\cat A$ is discrete.

\subsection{Duality}

In string diagrams, the \emph{duality} is easily expressed by flipping diagrams upside down. 
As an example, you can obtain the definition of \emph{coproducts} from that diagram flipped:
\In{align*}{
    \Compose\strdgathered\strdinv{
        \strdmid,\strdrlabel{\coprod F}
        ,\strdcell{h}
        ,\strdmid,\strdrlabel{B}
    } =
    \Compose\strdgathered\strdinv{
        \strdmid,\strdrlabel{\textstyle\coprod F}
        ,\strdbox{1.7}{1.7}{
            \strdfork{
                \strdmid
            }
            ,\strdtubestart
            ,\strdtube\strdtubemid
            ,\strdwidecell{1}{1}{v}
        },\strdat{Left StartCorner}{\strdoverlabel{\textstyle\coprod}}
       ,\strdmid,\strdrlabel{B}
    } \iff
    \Compose\strdgathered\strdinv{
        \strdfork{
            \strdmid
        }
        ,\strdtubestartline
        ,\strdtube\strdtubemid
        ,\strdwidecell{1}{1}{\operatorname{in}}
        ,\strdmid,\strdrlabel{\textstyle\coprod F}
        ,\strdcell{h}
        ,\strdmid,\strdrlabel{B}
    }
    =
    \Compose\strdgathered\strdinv{
        \strdfork{
            \strdmid
        }
        ,\strdtubestartline
        ,\strdtube\strdtubemid
        ,\strdwidecell{1}{1}{v}
        ,\strdmid,\strdrlabel{B}
    }
}
Note that the variant makes a domain opposite, while the dual does for a codomain.
