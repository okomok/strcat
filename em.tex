\subsection{EM Categories}

\strdexternalname{em}


\definitionof{Monad Algebra}{
    Given a monad $(T,\eta,\mu)$ on \cat C, a \emph{monad algebra}, denoted as $T$-algebra, consists of
    \In{enumerate}{
        \item an object $A \in \cat C$
        \item a morphism $\alpha : TA \to A$
    }
    satisfying the coherence
    \In{enumerate}{
        \item \emph{associativity}: $\alpha \circ \mu = \alpha \circ T(\alpha)$
        \item \emph{unitality}: $\alpha \circ \eta = \operatorname{id}$
    }
}

A $T$-algebra is depicted as
\In{align*}{
    \strdexternal\strdgathered{
        \strdmid,\strdrlabel{A}
        ,\strdwidecell{1}{1}{\alpha}
        ,\strdfork{\strdmid,\strdrlabel{A}}
        ,\strdtubestart
        ,\strdtube\strdtubemid
        ,\strdtubeendline
    }
}

The coherence can be depicted as
\In{align*}{
    \strdexternal\strdgathered{
        \strdmid
        ,\strdfork{\strdmid,\strdmid}
        ,\strdwidecell{1}{1}{\alpha}
        ,\strdtubestart
        ,\strdtube{
            \strdtubemid
            ,\strdtwincell{\mu}
            ,\strdfork{\strdtubemid}
            ,\strdtubecap{1}{1}
            ,\strdtubeendline
        }
    } &= 
    \strdexternal\strdgathered{
        \strdmid
        ,\strdwidecell{2}{2}{\alpha}
        ,\strdfork{
            \strdmid
            ,\strdfork{
                \strdmid
            }
            ,\strdwidecell{1}{1}{\alpha}
            ,\strdtubestart
            ,\strdtube\strdtubemid
        }
        ,\strdtube{
            \strdtubestart
            ,\strdtubeextend{1}{1}
            ,\strdtubemid
            ,\strdtubemid
            ,\strdtubeendline
        }
    } \\ 
    \strdexternal\strdgathered{
        \strdmid
        ,\strdfork{\strdmid,\strdmid}
        ,\strdwidecell{1}{1}{\alpha}
        ,\strdtubestart
        ,\strdtube{
            \strdtubemid
            ,\strdtwincell{\eta}
        }
    } &= \,
    \strdexternal\strdgathered{
        \strdmid
    }
}


\definitionof{EM Category}{
    Given a monad $(T,\eta,\mu)$, the \emph{Eilenberg-Moore(EM) category} of $T$, denoted as $\cat C^T$, %
    is a category whose objects are $T$-algebras and whose morphisms are %
    those of the form $h : A \to B$ such that
    \In{align*}{
        h \circ \alpha = \beta \circ T(h)
    }
    where $(A,\alpha)$ and $(B,\beta)$ are $T$-algebras.
}
This condition is depicted as
\In{align*}{
    \strdexternal\strdgathered{
        \strdmid,\strdrlabel{B}
        ,\strdcell{h}
        ,\strdmid,\strdrlabel{A}
        ,\strdwidecell{1}{1}{\alpha}
        ,\strdfork{\strdmid,\strdrlabel{A}}
        ,\strdtubestart
        ,\strdtube\strdtubemid
        ,\strdtubeendline
    } =
    \strdexternal\strdgathered{
        \strdmid,\strdrlabel{B}
        ,\strdfork{
            \strdmid,\strdrlabel{B}
            ,\strdcell{h}
            ,\strdmid,\strdrlabel{A}
        }
        ,\strdwidecell{1}{1}{\beta}
        ,\strdtubestart
        ,\strdtube{
            \strdtubemid
            ,\strdtubemid
            ,\strdtubeendline
        }
    }
}
A morphism in $\cat C^T$ is by compromise depicted as
\In{align*}{
    \strdexternal\strdgathered{
        \strdbox{1.5}{1.5}{
            \strdmid,\strdrlabel{B}
            ,\strdfork{\strdmid,\strdrlabel{B}}
            ,\strdwidecell{1}{1}{\beta}
            ,\strdtubestart
            ,\strdtube\strdtubemid
        }
        ,\strdyhalf\strdmid
        ,\strdcell{h}
        ,\strdyhalf\strdmid
        ,\strdbox{1.5}{1.5}{
            \strdmid,\strdrlabel{A}
            ,\strdfork{\strdmid,\strdrlabel{A}}
            ,\strdwidecell{1}{1}{\alpha}
            ,\strdtubestart
            ,\strdtube\strdtubemid
        }
    }
}
Boxes are objects. The composition can be depicted as
\In{align*}{
    \strdexternal\strdgathered{
        \strdbox{1.5}{1.5}{
            \strdmid,\strdrlabel{C}
            ,\strdfork{\strdmid,\strdrlabel{C}}
            ,\strdwidecell{1}{1}{\gamma}
            ,\strdtubestart
            ,\strdtube\strdtubemid
        }
        ,\strdyhalf\strdmid
        ,\strdcell{k}
        ,\strdyhalf\strdmid
        ,\strdbox{1.5}{1.5}{
            \strdmid,\strdrlabel{B}
            ,\strdfork{\strdmid,\strdrlabel{B}}
            ,\strdwidecell{1}{1}{\beta}
            ,\strdtubestart
            ,\strdtube\strdtubemid
        }
        ,\strdyhalf\strdmid
        ,\strdcell{h}
        ,\strdyhalf\strdmid
        ,\strdbox{1.5}{1.5}{
            \strdmid,\strdrlabel{A}
            ,\strdfork{\strdmid,\strdrlabel{A}}
            ,\strdwidecell{1}{1}{\alpha}
            ,\strdtubestart
            ,\strdtube\strdtubemid
        }
    } =
    \strdexternal\strdgathered{
        \strdbox{1.5}{1.5}{
            \strdmid,\strdrlabel{C}
            ,\strdfork{\strdmid,\strdrlabel{C}}
            ,\strdwidecell{1}{1}{\gamma}
            ,\strdtubestart
            ,\strdtube\strdtubemid
        }
        ,\strdyhalf\strdmid
        ,\strdcell{k}
        ,\strdmid
        ,\strdcell{h}
        ,\strdyhalf\strdmid
        ,\strdbox{1.5}{1.5}{
            \strdmid,\strdrlabel{A}
            ,\strdfork{\strdmid,\strdrlabel{A}}
            ,\strdwidecell{1}{1}{\alpha}
            ,\strdtubestart
            ,\strdtube\strdtubemid
        }
    }
}
A diagram for an identity morphism is left as an exercise.




\definitionof{EM Adjunction}{
    Given a monad $(T,\eta,\mu)$ on \cat C, define a functor $M : \cat C \to \cat C^T$ as
    \In{align*}{
        \strdexternal\strdgathered{
            \strdbox{2.3}{2.3}{
                \strdfork{\strdmid,\strdrlabel{B},\strdmid}
                ,\strdtube{
                    \strdtubestart
                    ,\strdtubemid
                    ,\strdtwincell{\mu}
                    ,\strdfork{
                        \strdtubemid
                    }
                    ,\strdtubecap{1}{1}
                }
            }
            ,\strdfork{
                \strdmid
                ,\strdat{Mid}{\strdcell{\catph{?}}}
            }
            ,\strdtubestart
            ,\strdtube\strdtubemid
            ,\strdbox{2.3}{2.3}{
                \strdfork{\strdmid,\strdmid,\strdrlabel{A}}
                ,\strdtube{
                    \strdtubestart
                    ,\strdtubemid
                    ,\strdtwincell{\mu}
                    ,\strdfork{
                        \strdtubemid
                    }
                    ,\strdtubecap{1}{1}
                }
            }
        }
    }
    a functor $U : \cat C^T \to \cat C$ as
    \In{align*}{
        \strdexternal\strdgathered{
            \strdbox{1.5}{1.5}{
                \strdmid,\strdrlabel{B}
                ,\strdfork{\strdmid,\strdrlabel{B}}
                ,\strdwidecell{1}{1}{\beta}
                ,\strdtubestart
                ,\strdtube\strdtubemid
            }
            ,\strdyhalf\strdmid
            ,\strdcell{h}
            ,\strdyhalf\strdmid
            ,\strdbox{1.5}{1.5}{
                \strdmid,\strdrlabel{A}
                ,\strdfork{\strdmid,\strdrlabel{A}}
                ,\strdwidecell{1}{1}{\alpha}
                ,\strdtubestart
                ,\strdtube\strdtubemid
            }
        } \mapsto
        \strdexternal\strdgathered{
            \strdmid,\strdrlabel{B}
            ,\strdcell{h}
            ,\strdmid,\strdrlabel{A}
        }
    }
    They constitute the \emph{EM adjunction} $M \dashv U$ whose adjunct is defined by %
    \In{align*}{
        \mylhs{\cat C^T(MA,(B,\beta))} &\cong \myrhs{\cat C(A,U(B,\beta))} \\
        \strdexternal\strdgathered{
            \strdbox{1.5}{1.5}{
                \strdmid,\strdrlabel{B}
                ,\strdfork{\strdmid,\strdrlabel{B}}
                ,\strdwidecell{1}{1}{\beta}
                ,\strdtubestart
                ,\strdtube\strdtubemid
            }
            ,\strdyhalf\strdmid
            ,\strdwidecell{1}{1}{h}
            ,\strdyhalf{
                \strdfork\strdmid
                ,\strdtubestart
                ,\strdtube\strdtubemid
            }
            ,\strdbox{2.3}{2.3}{
                \strdfork{\strdmid,\strdmid,\strdrlabel{A}}
                ,\strdtube{
                    \strdtubestart
                    ,\strdtubemid
                    ,\strdtwincell{\mu}
                    ,\strdfork{
                        \strdtubemid
                    }
                    ,\strdtubecap{1}{1}
                }
            }
        } & \mapsto
        \strdexternal\strdgathered{
            \strdmid,\strdrlabel{B}
            ,\strdwidecell{1}{1}{h}
            ,\strdfork{\strdmid,\strdmid,\strdrlabel{A}}
            ,\strdtubestart
            ,\strdtube\strdtubemid
            ,\strdtwincell{\eta}
        } \\
        \strdexternal\strdgathered{
            \strdbox{1.5}{1.5}{
                \strdmid,\strdrlabel{B}
                ,\strdfork{\strdmid,\strdrlabel{B}}
                ,\strdwidecell{1}{1}{\beta}
                ,\strdtubestart
                ,\strdtube\strdtubemid
            }
            ,\strdyhalf\strdmid
            ,\strdwidecell{1}{1}{\beta}
            ,\strdfork{
                \strdmid,\strdrlabel{B}
                ,\strdcell{f}
                ,\strdyhalf\strdmid
            }
            ,\strdtubestart
            ,\strdtube{\strdtubemid,\strdyhalf\strdtubemid}
            ,\strdbox{2.3}{2.3}{
                \strdfork{\strdmid,\strdmid,\strdrlabel{A}}
                ,\strdtube{
                    \strdtubestart
                    ,\strdtubemid
                    ,\strdtwincell{\mu}
                    ,\strdfork{
                        \strdtubemid
                    }
                    ,\strdtubecap{1}{1}
                }
            }
        } &\mymapsfrom
        \strdexternalz\strdgathered{
            \strdmid,\strdrlabel{B}
            ,\strdcell{f}
            ,\strdmid,\strdrlabel{A}
        }
    }
    This adjunction is $T$-associated.
}
