\section{Preliminary}

\subsection{Vacuous Truth}
todo

\subsection{Uniqueness Quantification}

For a boolean-valued function $P$, define:
\In{align*}{
    !aP(a) & \mydefeq P(a) \wedge \forall a' (P(a') \implies a = a')
}
You can state "there exists a unique $a$ such that $P$" using:
\In{align*}{
    \exists!aP(a) & \mydefeq \exists a ! a P(a) 
}
On the other hand,
\In{align*}{
    \exists a \big((!a P(a)) \wedge Q(a)\big)
}
means "there exists a unique $a$ such that $P$, moreover the $a$ is $Q$".

\subsection{Universality}

Category theory introduces many \textit{universal properties} like that 
\In{align*}{
    (\forall x \in X)(\exists!a\in A)(P(x,a))
}
, which is equivalent to an easier one:
\In{align*}{
    (\exists f : X \to A)(\forall x \in X)(\forall a \in A)(P(x,a) \iff a = f(x))
}
Find such a function $f$, that's the category theory.


\subsection{Lambda Expressions}

Following famous symbols like $\Sigma$, define:
\In{align*}{
    \Lambda_x y \mydefeq x \mapsto y 
}
for anonymous functions.

\subsubsection{Placeholder expressions}

For simple lambda expressions, you may use \textit{placeholders}:
\In{align*}{
    \texttt{?}{+}1 \mydefeq \Lambda_n n{+}1
}
Placeholder symbols can vary: $\texttt{?}$, $\texttt{-}$, $\texttt{=}$, etc.


\subsection{Family expressions}

Syntax of function-calls is world-standard and fixed:
\In{align*}{
    f(x) \text{ or } fx
}
, but sometimes you might want cuter syntax like that
\In{align*}{
    \strdgathered{\strdcellof{strd cellbase,sharp corners,regular polygon,regular polygon sides=5}{x}}
}
\textit{Family expressions} enable us to define a function with arbitary call syntax:
\In{align*}{
    (\strdgathered{\strdcellof{strd cellbase,sharp corners,regular polygon,regular polygon sides=5}{x}} \in Y) _ {x \in X}
}

\subsubsection{Dependent Functions}

Family expressions can do more. Let $F$ a set-valued function.
\In{align*}{
    (f (x) \in F(x)) _ {x \in X}
}
defines a function:
\In{align*}{
    f : X \to \textstyle\bigcup _ {x \in X} F(x)
}
such that
\In{align*}{
    (\forall x \in X)(f(x) \in F(x))
}
Such $f$ is called a \textit{dependent function}, for the $F(x)$ depends on a $x$. 
In case $F$ is a constant function which ignores $x$:
\In{align*}{
    (f(x) \in Y)_{x \in X}
}
, $f$ is a normal function $X \to Y$. 


\subsection{Coherence}

You write:
\In{align*}{
    3+1+2
}
rather than
\In{align*}{
    (3+(0+1))+2
}
, because you know the arithmetic laws:
\In{align*}{
    x+(y+z) = (x+y)+z \\
    0+x = x = x+0
}
Informally a set of laws which offers simpler syntax without ambiguity is called \textit{coherence condition} or \textit{coherence}.

